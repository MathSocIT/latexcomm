\documentclass[a4paper, 11pt]{exam}
\usepackage[utf8]{inputenc}
\usepackage{amsmath,amssymb,amsthm}
\usepackage{amsfonts}
\usepackage{float}
\usepackage{parskip}
\usepackage{array}
\usepackage{tabularx}
\usepackage[headheight=15pt, inner=2cm, outer=2cm, top=2cm, bottom = 2.2cm]{geometry}
\usepackage{setspace}
\usepackage{mathtools}
\usepackage{accents}
\usepackage{physics}
\usepackage{enumitem}


\title{\vspace{-0.5cm}MA1100(T) - Basic Discrete Mathematics (T) Suggested Solutions}
\author{Written by: Poon Chun Wai Victor\\Audited by: Daryl Chew}
\date{(Semester 1 : AY2023/24)}

\renewcommand{\partlabel}{\textbf{(\thepartno)}}
\renewcommand{\subpartlabel}{\textbf{(\thesubpart)}}
\renewcommand{\thepartno}{\alph{partno}}

\newcommand{\func}[3][]{\text{#2}^{\;#1}\!\left(#3\right)}
\newcommand{\solbox}[1]{\begin{EnvFullwidth}\begin{solution}#1\end{solution}\end{EnvFullwidth}}
\renewcommand{\solutiontitle}{\noindent\textbf{Solution:}\par\noindent}
\newcommand{\qedeqn}{\tag*{\qedsymbol}}
\renewcommand{\qedsymbol}{$\square$}

\addpoints
\setstretch{1.15}
\printanswers

\headrule
\footrule
\header{MA1100T Final}{}{Sem 1 AY 23/24}
\footer{}{}{Page \thepage}

\begin{document}
\maketitle
\thispagestyle{headandfoot}
\begin{questions}
\question[4] Suppose \(a\), \(b\) and \(n\) are positive integers. Suppose \(\gcd(a,b)\) is a prime, say \(q\), and that \(ab=n^2\). Prove that there exists \textbf{integers} \(c\) and \(d\) such that
\[\frac{a}{q}=c^2\quad\text{ and }\quad\frac{b}{q}=d^2.\]

\solbox{
We can construct \(c\) and \(d\) by
\[e_c(p)=\left\{\renewcommand{\arraystretch}{2}\begin{array}{ll}
    \dfrac{1}{2}e_a(p),&p\neq q,\\
    \dfrac{1}{2}(e_a(p)-1),&p=q,
\end{array}\right.
\quad\text{and}\quad
e_d(p)=\left\{\renewcommand{\arraystretch}{2}\begin{array}{ll}
    \dfrac{1}{2}e_b(p),&p\neq q,\\
    \dfrac{1}{2}(e_b(p)-1),&p=q,
\end{array}\right.
\]
First, since \(q\) is a prime, \(e_q(p)=0\) for all primes \(p\neq q\), and \(e_q(q)=1\).

Hence, \(e_{c^2q}=2e_c+e_q=e_a\) and \(e_{d^2q}=2e_c+e_q=e_a\), so \(\dfrac{a}{q}=c^2\) and \(\dfrac{b}{q}=d^2\).

It remains to show that \(e_c\) and \(e_d\) are well defined.

Since \(ab=n^2\), we have \(e_a(p)+e_b(p)=2e_n(p)\) for all primes \(p\). We consider two cases:

Case 1: \(p=q\).

Since \(\gcd(a,b)=q\), either \(1=e_a(q)\leq e_b(q)\) or \(1=e_b(q)\leq e_a(q)\).

In either case, since \(e_a(p)+e_b(p)=2e_n(p)\), both \(e_a(q)\) and \(e_b(q)\) are positive odd integers.

Hence, \(e_c(q)=\dfrac{1}{2}(e_a(q)-1)\) and \(e_d(q)=\dfrac{1}{2}(e_a(q)-1)\) are in \(\mathbb{N}\).

Case 2: \(p\neq q\).

Since \(\gcd(a,b)=q\), at least one of \(e_a(p)\) and \(e_b(p)\) is \(0\).

if \(e_a(p)=0\), we have \(e_b(p)=2e_n(p)\), so \(e_c(p)=0\) and \(e_d(p)=e_n(p)\).

If \(e_b(p)=0\), we have \(e_a(p)=2e_n(p)\), so \(e_c(p)=e_n(p)\) and \(e_c(p)=0\).

Hence, for all primes \(p\), \(e_c(p)\in\mathbb{N}\) and \(e_d(p)\) are in \(\mathbb{N}\).

Furthermore, \(e_c(p)\leq e_a(p)\) and \(e_d(p)\leq e_a(p)\). Since \(e_a,\,e_b\) have finite support, so do \(e_c\) and \(e_d\).

Hence, \(e_c\) and \(e_d\) are well defined. \qed
}
\clearpage
\question[3] Suppose \(a,b\in\mathbb{N}^+\). Prove that \(\gcd(a,b)=1\) if and only if \(\gcd(ab,a+b)=1\).
\solbox{
Lemma: Suppose \(x,\,y\in\mathbb{N}^+\). If there exists \(m,\,n\in\mathbb{Z}\) such that \(mx+ny=1\), then \(\gcd(x,y)=1\). 

Proof of Lemma:

Let \(\gcd(x,y)=d\). Then, \(d\mid x\) and \(d\mid y\), so \(d\mid mx+ny=1\).

Hence, \(d\leq 1\). Since \(d\geq 1\) by definition, \(d=1\). \qed

Proof of question:

(\(\Rightarrow\)) Suppose \(\gcd(a,b)=1\). 

Then, by Bezout's Identity, there exists \(h,k\in\mathbb{Z}\) such that \(ha+kb=1\). 

Then, we have the following equalities:
\begin{align*}
    1&=ha+kb\\
    &=(ha+kb)^2\\
    &=h^2a^2+2hkab+k^2b^2\\
    &=h^2a^2+h^2ab+k^2ab+k^2b^2+2hkab-h^2ab-k^2ab\\
    &=(h^2a+k^2b)(a+b)+(2hk-h^2-k^2)ab
\end{align*}
Hence, by the Lemma, \(\gcd(ab,a+b)=1\).

\((\Leftarrow)\) Suppose \(\gcd(ab,a+b)=1\). 

Then, by Bezout's Identity, there exists \(p,q\in\mathbb{Z}\) such that \(pab+q(a+b)=1\).

Then, we have the following equalities:
\begin{align*}
    1&=pab+q(a+b)\\
    &=(pb+q)a+qb
\end{align*}
Hence, by the Lemma, \(\gcd(a,b)=1\). \qed
}
\clearpage
\question Define a relation \(\sim\) on \(\mathbb{R}\) such that \(x\sim y\) if \(x-y\in\mathbb{Q}\).
\begin{parts}
    \part[3] Prove that \(\sim\) is an equivalence relation on \(\mathbb{R}\).
    \solbox{
    \underline{Reflexivity:}

    For all \(x\in\mathbb{R}\), \(x-x=0\in\mathbb{Q}\). Hence, \(x\sim x\) for all \(x\in\mathbb{R}\), and \(\sim\) is reflexive.

    \underline{Symmetric:}

    Suppose \(x,y\in\mathbb{R}\) are such that \(x\sim y\). 
    
    Then, \(x-y\in\mathbb{Q}\), so \(y-x=-(x-y)\in\mathbb{Q}\), so \(y\sim x\). Hence, \(\sim\) is symmetric.

    \underline{Transitivity:}

    Suppose \(x,y,z\in\mathbb{R}\) are such that \(x\sim y\) and \(y\sim z\).

    Then, \(x-y\) and \(y-z\) are in \(\mathbb{Q}\).

    Hence, \(x-z=(x-y)+(y-z)\in\mathbb{Q}\), so \(x\sim z\). Hence, \(\sim\) is transitive. \qed
    }
    \part[2] Prove that the quotient set \(\mathbb{R}/{\sim}\) is infinite.
    \solbox{
    Consider the set \(S=\{[\sqrt{2}k]_\sim:k\in\mathbb{Z}\}\).

    Note that the mapping \(f:\mathbb{Z}\to S\) by \(k\mapsto[\sqrt{2}k]_\sim\) is injective:

    If \([\sqrt{2}k]_\sim=[\sqrt{2}k']_\sim\), we have \(\sqrt{2}k\sim\sqrt{2}k'\).

    Hence, \(\sqrt{2}k-\sqrt{2}k'=(k-k')\sqrt{2}\in\mathbb{Q}\). 
    
    Since \(k-k'\) is an integer and \(\sqrt{2}\) is irrational, for \((k-k')\sqrt{2}\in\mathbb{Q}\) to hold, \(k-k'=0\).

    Hence, \(k=k'\).

    Since \(f\) injects from \(\mathbb{Z}\) to \(S\), and \(\mathbb{Z}\) is infinite, \(S\) is also infinite.

    Then, \(S\subseteq\mathbb{R}/\!\sim\), so \(\mathbb{R}/\!\sim\) is infinite. \qed
    }
\end{parts}
\clearpage
\question[4] Suppose \(J\) is a nonempty indexing set. Let \((A_j)_{j\in J}\) be a family of sets, and define their \textit{disjoint union} as
\[\bigsqcup_{j\in J}A_j=\{(j,a):j\in J, a\in A_j\}.\]
For each \(j\in J\), define the function \(i_j:A_j\to\bigsqcup_{j\in J}A_j\) by \(a\mapsto (j,a)\). Define also the family of functions \((f_j:A_j\to X)_{j\in J}\) (consisting of one such \(f_j\) for each \(j\in J\)). Prove that there exists a unique function \(f:\bigsqcup_{j\in J}A_j\to X\) such that \(f_j=f\circ i_j\) for each \(j\in J\).
\solbox{
Define \(f\) by \((j,a)\mapsto f_j(a)\). We first prove \(f\) is well-defined.

\(f(j,a)\) always exists, because the family of functions has one \(f_j\) for every \(j\in J\).

Then, Suppose \((j_1,a_1)=(j_2,a_2)\). Then, \(j_1=j_2\), so \(f_{j_1}=f_{j_2}\), and \(a_1=a_2\).

Since all the \(f_j\)'s are well defined functions, and \(a_1=a_2\), we have
\[f(j_1,a_1)=f_{j_1}(a_1)=f_{j_1}(a_2)=f_{j_2}(a_2)=f(j_2,a_2)\]
so \(f\) is well-defined.

Then, for each \(j\in J\), we have that
\[f\circ i_j(a)=f(i_j(a))=f(j,a)=f_j(a)\]
for all \(a\in A_j\), so \(f_j=f\circ i_j\).

To show \(f\) is unique, suppose \(g:\bigsqcup_{j\in J}A_j\to X\) is a function such that \(f_j=g\circ i_j\) for each \(j\in J\).

Note that \(i_j\) is onto, as for each \((j,a)\in \bigsqcup_{j\in J}A_j\), \(i_j(a)=(j,a)\).

Then we have the following equalities that hold for all \((j,a)\in\bigsqcup_{j\in J}A_j\):
\begin{align*}
    g(j,a)&=g(i_j(a))\quad[\text{because } i_j\text{ is onto}]\\
    &=f_j(a)\\
    &=f(i_j(a))\\
    &=f(j,a)
\end{align*}
so \(g=f\) as desired. \qed
}
\clearpage
\question[3] Suppose \(m,n\in\mathbb{N}^+\). We attempt to define a function \(f:[mn]\to[m]\times[n]\) by
\[f(R_{mn}(a))=(R_m(a),R_n(a)).\]
Prove that \(f\) is well-defined.
\solbox{
Suppose \(c\in[mn]\). Since the \(R_b\) function is onto for all \(b\in\mathbb{N}^+\), there exists some integer \(a\) such that \(c=R_{mn}(a)\). Hence, \(R_m(a)\) and \(R_n(a)\) exist, so \(f(c)\) exists.

Suppose now that \(R_{mn}(a_1)=c=R_{mn}(a_2)\).

Then, \(a_1=q_1(mn)+c\) and \(a_2=q_2(mn)+c\), for some \(q_1,q_2\in\mathbb{Z}\).

Then, \(a_1-a_2=(q_1-q_2)mn\), so \(m\mid a_1-a_2\). Hence, \(R_m(a_1)=R_m(a_2)\).

Likewise, \(n\mid a_1-a_2\), so \(R_n(a_1)=R_n(a_2)\).

Hence, \((R_m(a_1),R_n(a_1))=(R_m(a_2),R_n(a_2))\), and \(f(c)\) has a unique value. \qed
}
\clearpage 
\question Let \(C\) be the set of functions from \(\mathbb{N}\) to \(\{0,1\}\) that are \textit{eventually constant}, that is, for each \(f\in C\), there exists \(n\in\mathbb{N}\) such that for all \(m>n\), \(f(m)=f(n)\). For each \(f\in C\), define the set \(S_f\) by
\[S_f=\{n\in\mathbb{N}:(\forall m>n)(f(m)=f(n))\}.\]
Define the function \(F:C\to\{0,1\}^{<\mathbb{N}}\) by
\[F(f)=(f(0),f(1),\dots,f(\min(S_f))).\]
\begin{parts}
    \part[3] Prove that \(F\) is one-to-one.
    \solbox{
    Suppose \(F(f_1)=F(f_2)\). Then, by the definition of \(F\), we have
    \[(f_1(0),f_1(1),\dots,f_1(\min(S_{f_1})))=(f_2(0),f_2(1),\dots,f_2(\min(S_{f_2}))).\]
    For the two sequences to be equal, they must have equal length. Hence, \(\min(S_{f_1})=\min(S_{f_2})\).

    Let \(\min(S_{f_1})=\min(S_{f_2})=k\). For the two sequences to be equal, they are also termwise equal. Hence, for all \(m\leq k\), \(f_1(m)=f_2(m)\).

    By the definition of \(S_f\), for all \(m>k\), \(f(m)=f(k)\). Hence, \(f_1(m)=f_1(k)=f_2(k)=f_2(m)\).

    Hence \(f_1(m)=f_2(m)\) for all \(m\in\mathbb{N}\), so \(f_1=f_2\), as desired. \qed
    }
    \part[1] Prove that \(C\) is countable.
    \solbox{
    Since \(\{0,1\}\) is finite, \(\{0,1\}^{<\mathbb{N}}\) is countably infinite. 
    
    Hence, there exists a bijection \(g\) from \(\{0,1\}^{<\mathbb{N}}\) to \(\mathbb{N}\).

    Then, \(g\circ F\) is an injection from \(C\) to \(\mathbb{N}\), so \(C\) is countable. \qed
    }
\end{parts}
\clearpage
\question[3] Prove that for every pair of real numbers \(q<r\), there exists an irrational number that is strictly between them.
\solbox{
Since \(\mathbb{Q}\) is dense in \(\mathbb{R}\), there is a rational number \(x\) strictly between \(q\) and \(r\).

Then, by the same argument, there is a rational number \(y\) strictly between \(x\) and \(r\).

Then we know there is an irrational number strictly between two rational numbers, so we are done, as there is an irrational \(z\) such that \(q<x<z<y<r\). \qed
}
\solbox{
(Alternative)

Let \(x<y\) be a pair of real numbers such that \(q=\sqrt{2}x\) and \(r=\sqrt{2}y\).

Since \(\mathbb{Q}\) is dense in \(\mathbb{R}\), there exists some rational \(z\) that is strictly between \(x\) and \(y\).

Furthermore, we can take \(z\) to be non-zero. If \(z\) was \(0\), we can take some rational \(z'\) that is strictly between \(z\) and \(y\), since \(\mathbb{Q}\) is dense in \(\mathbb{R}\). Since \(z<z'\), \(z'\) would be non-zero.

Hence, we have:
\[x<z<y\implies\sqrt{2}x<\sqrt{2}z<\sqrt{2}y\implies q<\sqrt{2}z<r
\]
where \(\sqrt{2}z\) is a product of an irrational and a non-zero rational, and is hence irrational. \qed
}
\end{questions}

\end{document}
