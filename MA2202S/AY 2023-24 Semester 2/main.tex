\documentclass[12pt]{article}

\usepackage{amsmath}
\usepackage{amssymb}
\usepackage[margin=1in]{geometry}
\usepackage{xcolor}
\usepackage{graphicx,tipa}% http://ctan.org/pkg/{graphicx,tipa}
\usepackage{pgfplots}
\usetikzlibrary{intersections, fillbetween} % Ensure this library is included
\usepgfplotslibrary{fillbetween} % Necessary for fill between feature

\pgfplotsset{compat=1.17}

\title{MA2202S - Algebra I (S) Suggested Solutions}
\author{(Semester 2, AY2023/2024)}
\date{Written by: Sarji Bona\\Audited by: Thang Pang Ern}

\begin{document}
\maketitle 

\section*{Question 1}
Let $G$ be a group. Assume $G$ has a unique subgroup of order $n$, for some positive integer $n$. Denote this subgroup by $H$. Prove that $H$ is a normal subgroup of $G$. \\

\noindent \textbf{Solution:} Let $g \in G$. Note that \[\phi:G\to gHg^{-1}\quad\text{where}\quad h\mapsto ghg^{-1}\quad\text{is bijective},\]
so $|gHg^{-1}| = |H| = n$. But $|H|$ is the unique subgroup of order $n$, so $gHg^{-1} = H$, which implies $H$ is a normal subgroup of $G$. \hfill $\Box$
    
\section*{Question 2}
Let $p$ be a prime. Prove that $a^p \equiv a \pmod{p}$ for any $a \in \mathbb{Z}$\footnote{This is known as Fermat's little theorem}. \\

\noindent \textbf{Solution:} Consider the multiplicative group $(\mathbb{Z}/p\mathbb{Z})^{\times} = \{1, 2, \dots, p - 1\}$. For each $a \in (\mathbb{Z}/p\mathbb{Z})^{\times}$, there exists a positive integer $k$ such that $a^k = 1$. By Lagrange's theorem, $k \mid \left(p - 1\right)$, so $p - 1 = km$ for some integer $m$. So $a^{p-1} = a^{km} = 1 \Rightarrow a^p = a$. Hence, $a^p \equiv a \pmod{p}$ for any $a \in \mathbb{Z}$. \hfill $\Box$

\section*{Question 3}
Let $G$ be a $p$-group for some prime $p$ with a normal subgroup $H$. Assume $H$ is of order $p$. Prove that $H$ is contained in the center of $G$. \\

\noindent \textbf{Solution:} Let the order of $G$ be $p^n$ for $n\in\mathbb{N}$. Note that $N_G(H)/C_G(H) = G/C_G(H)$ is isomorphic to a subgroup of $\mathrm{Aut}(H)$ which has order $p-1$. Since $p^n/|C_G(H)|$ divides $p - 1$,  $p^n/|C_G(H)|$ must be a power of $p$ less than or equal to $p-1$, implying \[p^n/|C_G(H)| = 1\quad\text{so}\quad |C_G(H)| = p^n = |G|\quad\text{so}\quad C_G(H) = G.\]
Hence, all elements of $H$ commute with all elements of $G$, thus, $H\subseteq Z\left(G\right)$. \hfill $\Box$
    
\section*{Question 4}
Let $G$ be a finite group. Let $H$ and $K$ be two subgroups of $G$ such that $|H|$ and $|K|$ are coprime. We consider the natural (left) action of $H$ on $G/K$. Prove that all the orbits have the cardinality $|H|$. \\

\noindent \textbf{Solution:} If $|H| = 1$, all orbit sizes must divide $|H|$ by the orbit-stabilizer theorem, so they must be all $1$. If $|H| > 1$, let $|K| = n$, and $h_1,h_2$ be any two distinct elements of $H$, and $g\in G$. Then, \[h_1gK = h_2gK\quad\text{so}\quad g^{-1}h_2^{-1}h_1g \in K\quad \text{so}\quad (g^{-1}h_2^{-1}h_1g)^n = e\quad \text{so}\quad g^{-n}h_2^{-n}h_1^ng^n = e.\] 
Hence, $(h_2^{-1}h_1)^n = e$. So, the order of $h_2^{-1}h_1$ divides both $|K|$ and $|H|$, which can only be $1$, implying $h_1 = h_2$, a contradiction. Thus, for any $g \in G$, all $h_igK$ are pairwise distinct for all $h_i \in H$. Therefore, all orbits have cardinality $|H|$. \hfill $\Box$

\section*{Question 5}
Let $G$ be a group. Prove that if $|G| = 56$ then $G$ is not simple. \\

\noindent \textbf{Solution:} For a prime $p$, let $n_p$ be the number of distinct $p$-Sylow subgroups of $G$. By Sylow's third theorem, \[n_7\equiv 1\pmod 7\text{ and }n_7\mid 56\quad\text{so}\quad n_7=1\text{ or }8.\] 
If $n_7 = 1$, then it has a normal $7$-Sylow subgroup, implying $G$ is not simple by Sylow's second theorem. Next, assume $n_7 = 8$, then there is $1$ element of order $1$ and $8 \cdot (7 - 1) = 48$ elements of order $7$.
\newline
\newline By Sylow's second theorem, \[n_2\equiv 1\pmod 2\text{ and }n_2\mid 56\quad\text{so}\quad n_2=1\text{ or }7.\] 
If $n_2 = 1$, again $G$ is not simple by Sylow's second theorem. If $n_2 = 7$, then there are $7$ subgroups of order $8$. Let $H_1$ and $H_2$ be two distinct $2$-Sylow subgroups. Then $7$ out of the $8$ elements in $H_1$ have order $2, 4$, or $8$. There is also at least $1$ element in $H_2$ that is not in $H_1$, which has order $2, 4$, or $8$. In total, there are at least $1 + 48 + 7 + 1 = 57$ elements in $G$, a contradiction. Thus, if $|G| = 56$, then $G$ is not simple. \hfill $\Box$
\newpage
\section*{Question 6}
Let $G$ be a group. Prove that the diagonal subgroup $\Delta(G) = \{(g,g) \in G \times G\} \subset G \times G$ is normal in $G \times G$ if and only if $G$ is abelian. \\

\noindent \textbf{Solution:} If $G$ is abelian, then for any $(g_1, g_2) \in G \times G$ and $(g, g) \in \Delta(G)$, we have \[(g_1, g_2)(g, g)(g_1^{-1}, g_2^{-1}) = (g_1gg_1^{-1}, g_2gg_2^{-1}) = (g_1g_1^{-1}g, g_2g_2^{-1}g) = (g, g) \in \Delta(G),\] so $(g_1, g_2)\Delta(G)(g_1^{-1}, g_2^{-1}) \subseteq \Delta(G)$, implying $\Delta(G)\unlhd G \times G$. \\

\noindent We then prove the forward direction by contraposition. Suppose $G$ is not abelian, then there exist $a, b \in G$ such that $ab \ne ba \Rightarrow a \ne bab^{-1}$. So $(a, b)(a, a)(a^{-1}, b^{-1}) = (a, bab^{-1}) \notin \Delta(G)$ since $bab^{-1}\ne a$, implying that $\Delta(G)$ is not normal. \\

\noindent Therefore, $\Delta(G)$ is normal in $G \times G$ if and only if $G$ is abelian. \hfill $\Box$

\newpage

\section*{Question 7}
Let $G$ be a group with a subgroup $H$ of finite index. Assume $H$ is abelian and $[G : H] = n$. Let $S = \{t_1,\dots, t_n\} \subset G$ be a complete set of the representatives of the cosets $G/H$. The left
action of $G$ on $G/H$ induces a permutation on $S$, such that $gt_iH
= t_{g(i)}H$ for any $g \in G$. Here we abuse notations, and denote the induced permutation of $g$ by $g$ as well. \\

\noindent We define a transfer map $\tau : G \to H$ as follows. For any $g \in G$ and $t_i \in S$, we write $gt_i = t_{g(i)}h_{i,g}$ for some $h_{i,g} \in H$ and $t_{g(i)} \in S$. We define $\tau(g) = h_{1,g}\dots h_{n,g}$.

\begin{enumerate}
    \item[(a)] Show that $\tau$ is a group homomorphism. 
    \item[(b)] Show that $\tau$ is independent of the choice $S$ of representatives, as well as the ordering of elements in $S$.
\end{enumerate}
\noindent \textbf{Solution:} 
\begin{enumerate}
    \item[(a)] Note that for any $g \in G$ and $i \in \{1, 2, \dots, n\}$, $t_{g(i)}^{-1}gt_i \in H$. For any $g_1, g_2 \in G$, we have 
    \begin{align*}
        \tau(g_1)\tau(g_2) &= \prod_{i=1}^n t_{g_1(i)}^{-1}g_1t_i \cdot \prod_{j=1}^n t_{g_2(j)}^{-1}g_2t_j = \prod_{j=1}^n t_{g_1(g_2(j))}^{-1}g_1t_{g_2(j)} \cdot \prod_{j=1}^n t_{g_2(j)}^{-1}g_2t_j \\
        &= \prod_{j=1}^n t_{g_1(g_2(j))}^{-1}g_1t_{g_2(j)} t_{g_2(j)}^{-1}g_2t_j = \prod_{j=1}^n t_{g_1(g_2(j))}^{-1}g_1g_2t_j \\
        &= \tau(g_1g_2).
    \end{align*}
    Hence, $\tau$ is a group homomorphism. \hfill $\Box$
    \item[(b)] Let $S' = \{u_1, \dots, u_n\} \subset G$ be any complete set of the representatives of the cosets $G/H$. Define $\tau'$ similarly to $\tau$ but on $S'$ instead. There exists a unique permutation $\sigma \in S_n$ such that $t_iH = u_{\sigma(i)}H$ for all $i \in \{1, 2, \dots, n\}$. Then for all $i \in \{1, 2, \dots, n\}$, there exists $h_i \in H$ such that $t_ih_i = u_{\sigma(i)}$. \\

    Now let $g \in G$. Denote by $g'$ the induced permutation of $g$ on $S'$ where $gu_iH = u_{g'(i)}H$ for all $i \in \{1,2,\dots, n\}$. Then for all $i \in \{1, 2, \dots, n\}$, we have \[u_{g'(\sigma(i))}H = gu_{\sigma(i)}H = gt_iH = t_{g(i)}H = t_{g(i)}h_{g(i)}H = u_{\sigma(g(i))}H,\] so $g' \circ \sigma = \sigma \circ g$. Then 
    \begin{align*}
        \tau'(g) &= \prod_{i=1}^n u_{g'(i)}^{-1}gu_i = \prod_{i=1}^n u_{g'(\sigma(i))}^{-1}gu_{\sigma(i)} = \prod_{i=1}^n u_{\sigma(g(i))}^{-1}gt_ih_i = \prod_{i=1}^n (t_{g(i)}h_{g(i)})^{-1}gt_ih_i \\
        &= \prod_{i=1}^n h_{g(i)}^{-1}t_{g(i)}^{-1}gt_ih_i = \left(\prod_{i=1}^n t_{g(i)}^{-1}gt_i\right)\left(\prod_{i=1}^n h_{g(i)}^{-1}\right)\left(\prod_{i=1}^n h_i\right) = \prod_{i=1}^n t_{g(i)}^{-1}gt_i \\
        &= \tau(g).
    \end{align*}
    Therefore, $\tau$ is independent of $S$. \hfill $\Box$
\end{enumerate}
\end{document}
