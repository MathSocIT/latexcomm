\documentclass{article}
\usepackage{lmodern}
\usepackage{amssymb,amsmath}
\usepackage[margin=1in]{geometry}
\usepackage{enumitem}
\usepackage{tikz-cd}

\newcommand{\R}{\mathbb{R}}
\newcommand{\N}{\mathbb{N}}
\newcommand{\Z}{\mathbb{Z}}
\newcommand{\mf}{\mathfrak}
\newcommand{\PP}{\mathbb{P}}
\newcommand{\A}{\mathbb{A}}
\newcommand{\OO}{\mathcal{O}}

%\setlength\parindent{0pt}

\title{MA4273 - Algebraic Geometry of Curves and Surfaces \\Suggested Solutions}
\author{(Semester 2, AY2022/2023)}
\date{Written by: Chow Boon Wei}

\begin{document}\maketitle

\subsection*{Question 1}
We are given that $X$ is affine and $f\in \mathcal{O}(U)$.
So, for each $p\in U$, there exists an open neighborhood $V_p$ of $p$ and $g_p,h_p\in \mathcal{O}(X)$ 
such that $h_p\big|_{V_p} \in \mathcal{O}(V_p)^\times = \emptyset$ and $f=\frac{g_p}{h_p}$ on $U$. 
Now, $V_p$ is open in $U$ and so is open in $X$. Since $X$ is irreducible, $V_p$ is dense in $X$. 
Without loss of generality, we may assume that $\gcd(h_p,g_p)=1$. 
Let $q\in U$ be another point. Similarly, $V_q$ is also dense in $X$. So, $V_p \cap V_q \neq \emptyset$. 
Then, $f=\frac{g_p}{h_p}=\frac{g_q}{h_q}$ on $V_p \cap V_q$. 
Therefore, $h_qg_p=g_qh_p$ on $V_p \cap V_q$. Since $h_p$ does not divide $g_p$ and 
$h_q$ does not divide $g_q$, we see that $h_p$ divides $h_q$ and $h_q$ divides $h_p$. So, $h_p = \lambda h_q$ 
for some $\lambda \in k^\times$. This means that $h_p$ and $h_q$ have the same vanishing locus. So, 
$h_p\big|_{V_p \cup V_q} \in \mathcal{O}(V_p \cup V_q)^\times$ and 
$h_q\big|_{V_p \cup V_q} \in \mathcal{O}(V_p \cup V_q)^\times$.
A similar argument shows that we have $g_p = \lambda g_q$. Now, on $V_p \cup V_q$, we have 
$\frac{g_p}{h_p} = \frac{\lambda g_q}{\lambda h_q} = \frac{g_q}{h_q}$.
So, $f = \frac{g_p}{h_p} = \frac{g_p}{h_p}$ has the same ratio on $V_p \cup V_q$. 
Since $p,q$ is arbitrary, $f=\frac{g}{h}$ for some $g,h \in \mathcal{O}(X)$ with 
$h\big|_U \in \mathcal{O}(U)^\times$. 
\subsection*{Question 2}
The restriction map is a morphism of rings. So, we just need to show that it is also an isomorphism of 
rings. To do so, it suffices to show that it is bijective. 

First, we shall show surjectivity. 
Given an $f\in \mathcal{O}(U)$, the result of question $1$ tells us that $f=\frac{g}{h}$ for 
some $f,g\in \mathcal{O}(X)$ with $g\big|_U \in \mathcal{O}(U)^\times$. Hence, the zero set of $h$ 
is a subset of $Y$. Suppose for the sake of contradiction that $h \notin \mathcal{O}(X)^\times$.
Let $\ell$ be an irreducible factor of $h$. Then, the zero set of $\ell$ is 
contained in the zero set of $h$ which is in turn contained in $Y$. Now, this means that the 
zero set of $\ell$ has codimension at least $2$. But, this is not possible because the zero set 
of $\ell$ is a hypersurface and should have codimension $1$. So, we must have $h \in \mathcal{O}(X)^\times$. 
Thus, we can extend $f=\frac{g}{h}$ to a regular function $F=\frac{g}{h}$on $X$. 

Next, we shall show injective. Recall that $X$ is irreducible as a topological space. 
Since $U$ is open in $X$, we see that $U$ is also dense in $X$. Thus, this map is injective because any 
$f\in \mathcal{O}(U)$ is the restriction of a unique $F\in \mathcal{O}(X)$. 

\subsection*{Question 3}
Let $X$ be a pre-algebraic set such that $\OO(X)$ is a finitely generated $k$ algebra. 
Then, we have a $k$ algebra homomorphism $\psi: k[x_1,\dots,x_n] \twoheadrightarrow \OO(X)$ for some $n\in\N$.  
So, we have $\OO(X) \cong k[x_1,\dots,x_n] / I$ where $\ker(\psi) = I \subset k[x_1,\dots,x_n]$. 
Consider an open affine cover $\{U_i\}_{i\in I}$ of $X$. 
For each $i\in I$, we have a map $\OO(X) \to \OO(U_i)$ by restriction of the regular functions. 
This defines a morphism $\phi_i: U_i \to Z(I)$. 
Now, we will glue these morphisms together to get a morphism $X \to Z(I)$. 
Let $i,j\in I$ be fixed. Take any $F\in \OO(X)$. Now, restriction of $F$ to $U_i$ 
and then to $U_i \cap U_j$ is the same as restriction of $F$ to $U_j$ and then to 
$U_i \cap U_j$. Therefore, the maps $\phi_i$ and $\phi_j$ agree when restriction to 
the set $U_i \cap U_j$. Thus, the morphisms $\{\phi_i\}_{i\in I}$ glue to give a morphism 
$\phi: X \to Z(I)$. Give the affine algebraic set $Z(I)$ the name $X^{\textrm{aff}}$. 

Let $Y$ be another affine algebraic set and $f: X \to Y$ be a morphism. 
Then, any $G \in \OO(Y)$ gives rise to an $f_*(G) \in \OO(X)$. This gives us a map 
$\OO(Y) \to \OO(X)$ which induces a map $f^{\textrm{aff}}: X^{\textrm{aff}} \to Y$ 
since $Y$ and $X^{\textrm{aff}}$ is defined by $\OO(X)$. To see that the diagram 
commutes, we check locally. Let $U \subset X$ be any affine open subset of $X$. 
Then, $f^{\textrm{aff}} \circ \phi$ is given by pullback of a regular function from $\OO(Y)$ to $\OO(X)$ 
and then restricting to $\OO(U)$. But, this is the same as the pullback of a regular function by $f\big|_{U}$. 
Therefore, the morphisms $f^{\textrm{aff}} \circ \phi$ and $f$ agree on any affine open $U \subset X$. 
Hence, we have $f = f^{\textrm{aff}} \circ \phi$ which shows that the diagram commutes. 

Finally, the same argument shows that the morphism $f^{\textrm{aff}}$ is unique. 
To see this, let $G: X^{\textrm{aff}} \to Y$ be another such morphism making the diagram commute. 
Then, by restricting to any open affine subset $U\subset X$, we see that 
$G \circ \phi \big|_U = f \big|_U = f^{\textrm{aff}} \circ \phi \big|_U$. 
Therefore, both of these maps induce the same map $\OO(Y) \to \OO(U)$. This means that 
the restriction of $G_*: \OO(Y) \to \OO(X)$ and $f^{\textrm{aff}}_*: \OO(Y) \to \OO(X)$ 
are the same when restricted to any open affine $U\subset X$. So, we have $G=f^{\textrm{aff}}$ which 
gives us uniqueness. 

\subsection*{Question 4}
Let $X$ be a separated pre-algebraic set. Let $U,V$ be affine open subsets of $X$. 
We want to show that $U \cap V$ is also an affine open subset of $X$. 
It is clear that $U \cap V$ is open in $X$. We just have to show that it is affine. 
Since $X$ is separated, the image of the diagonal morphism $\Delta: X \to X \times X$ is closed. 
Therefore, the preimage of $U \times V$ under the diagonal morphism is $U \cap V$. 
Now, diagonal morphism restricted to $U \cap V$ gives an isomorphism onto its image $U \times V$. 
Therefore, it suffices to show that $U \times V$ is affine. But $U \times V$ is affine 
because the (fibered) product of affine sets is affine. 

\subsection*{Question 5}
Consider the affine plane with double origin. Let $U$ and $V$ the two copies of $\A^2_k$ in $X$. 
Then, it is clear that $U$ and $V$ are both affine open subsets. 
But, $U \cap V = \A^2_k \backslash \{(0,0)\}$ is not affine. 
%See corollary 3.2.24
 
\subsection*{Question 6}
Let $E=Z_h(y^2z-x^3+xz^2) \subset \PP^2_k$. 
Eisenstein's criterion applied to $$y^2z-x^3+xz^2 = -x^3+y^2z+xz^2 \in k[y,z][x] = k[x,y,z]$$ 
with the prime $z\in k[y,z]$ tells us that $y^2z-x^3+xz^2$ is irreducible. 
Then, we see that $E$ is a projective curve. To show that $E$ is nonsingular, 
we shall reduce to the affine case and use the Jacobian criterion. 
Consider the affine piece $U_1 = \PP^2_k \backslash Z_h(z)$. Then, we have 
$E \cap U_1 \cong Z(y^2-x^3+x) \subset \A^2_k$. Let $f = y^2-x^3+x \in k[x,y]$. Then, 
$$
J_f(x,y) 
= 
\begin{pmatrix}
    f_x(x,y)
    &f_y(x,y)
\end{pmatrix}
= 
\begin{pmatrix}
    -3x^2+1
    &2y 
\end{pmatrix}.
$$
If $y \neq 0$, the Jacobian has rank $1$ and so $(x,y)$ is not a singular point. 
If $y = 0$, then $-x^3+x=0$. So, $x\in \{-1,0,1\}$ which means $-3x^2+1\neq 0$. 
So, the Jacobian has rank $1$ and so $(x,y)$ is not a singular point. 
Therefore, $f$ has no singular points in $U_1$. The only other possible singular point is 
when $z=0$. But $z=0$ gives us $x=0$. So, the last point to check is $[0:1:0]$. 
We can check this by reducing to the open piece $U_2 = \PP^2_k \backslash Z_h(y)$. 
Then, we have 
$E \cap U_2 \cong Z(z-x^3+xz^2) \subset \A^2_k$. Let $g = z-x^3+xz^2 \in k[x,z]$. Then, 
$$
J_g(0,0) 
= 
\begin{pmatrix}
    g_x(0,0)
    &g_z(0,0)
\end{pmatrix}
= 
\begin{pmatrix}
    1
    &1 
\end{pmatrix}
$$ which has rank $1$. So, $[0:1:0]$ is not a singular point which completes the proof. 

\subsection*{Question 7}
Suppose for the sake of contradiction that $E$ is birational to $\PP^1_k$. 
Since both $E$ and $\PP^1_k$ are nonsingular projective curves, 
this isomorphism extends uniquely to an isomorphism. 
Take $x,y,z$ to be the coordinates of $\PP^2_k$. 
Then, $ E \cap U_3 = E \cap \PP^2_k \backslash Z(z) \cong Z(y^2-x^3+x) \subset \A^2_k$ is isomorphic 
to a subset $V$ of $\PP^1_k$. However, when $z=0$, we have $y^2z-x^3+xz^2=-x^3=0$. Therefore, 
$E \cap \PP^2_k \backslash Z(z) = E \backslash \{[0:1:0]\}$. This tells us that $V = \PP^1_k \backslash \{p\}$ 
for some $p \in \PP^1_k$. But, $\PP^1_k \backslash \{p\} \cong \A^1_k$. This puts us in the affine case. 
Thus, $Z(y^2-x^3+x)$ is isomorphic to $\A^1_k$. Hence, $\mathcal{O}(Z(y^2-x^3+x))$ and $\mathcal{O}(\A^1_k)$ 
are isomorphic as rings. But, $\mathcal{O}(Z(y^2-x^3+x)) = A(Z(y^2-x^3+x)) = k[x,y]/(y^2-x^3+x)$ is 
not a UFD and $\mathcal{O}(\A^1_k) = k[t]$ is a UFD. This is a contradiction. 

\subsection*{Question 8}
This is like the Veronese embedding.  
Suppose that $[x:y:z]=[a:b:c] \in PP^2_k$, the $x=\lambda a$, $y=\lambda b$ and $z=\lambda c$ for some 
$\lambda \in k^\times$. So,  
$$
x^2=\lambda^2a^2,xy=\lambda^2ab,xz=\lambda^2ac,y^2=\lambda^2b^2,yz=\lambda^2ac 
\textrm{ and }
z^2=\lambda^2c^2.
$$ Hence, $[x^2:xy:xz:y^2:yz:z^2]=[a^2:ab:ac:b^2:ac:c^2]$ which shows that $\phi$ is well-defined. 

To show that this is a morphism, it suffices to check on an open cover. 
Naturally, we may pick the open cover $U_1 = \PP^2_k \backslash Z_h(x)$, 
$U_2 = \PP^2_k \backslash Z_h(y)$ and $U_3 = \PP^2_k \backslash Z_h(z)$. 
On $U_1 \cong \A^2_k$, we have $\phi(x,y) = [x^2:xy:x:y^2:y:1]$. 
So, $\phi\big|_{U_1}$ is given by $6$ regular functions on $U_1 \cong \A^2_k$. 
It is clear that these morphisms do not vanish simultaneously. 
This tells us that $\phi\big|_{U_1}$ is a morphism $U_1 \to \PP^5_k$. 
Similar arguments show that $\phi\big|_{U_2}$ and $\phi\big|_{U_3}$ are morphisms $U_2 \to \PP^5_k$ 
and $U_3 \to \PP^5_k$. Therefore, $\phi: U \to \PP^5_k$ is a morphism. 

Let $\phi([x:y:z]) = [x^2:xy:xz:y^2:yz:z^2]=[a^2:ab:ac:b^2:ac:c^2] = \phi([a:b:c]) \in \PP^5_k$ be given. 
We cannot have $a=0,b=0$ and $c=0$ otherwise $[a:b:c]$ is not a point in $\PP^2_k$. 
So, we either have $a \neq 0,b \neq 0$ or $c \neq 0$. Without loss of generality, suppose that $a\neq 0$. 
Then, $a^2 \neq 0$. So, $x^2 = \lambda a^2$ for some $\lambda \in k^\times$. Hence, $\lambda = \frac{x^2}{a^2}$.
Thus, $x = \frac{x}{a} a = \sqrt{\lambda} a$. If $b=0$, we get that $y^2 = \lambda^2b^2 = 0$. So, $y=0$. 
If $b\neq 0$, then $y = \frac{y}{b} b = \sqrt{\lambda} b$. 
Similarly, if $c=0$, we get that $z^2 = \lambda^2c^2 = 0$. 
If $c\neq 0$, then $z = \frac{z}{c} c = \sqrt{\lambda} c$. 
In any case, this means that $[x:y:z] = [a:b:c]$ which gives us injectivity. 
\subsection*{Question 9} 
We identify $\PP^5_k$ with the closed set $Z_h(b-d,f-h,c-g)$ where 
$$
\begin{bmatrix}
    a &b  &c\\
    d  &e &f\\
    g  &h &i\\
\end{bmatrix} 
$$
are the variables of $\PP^9_k$. Under this identification, we have 
$$
\phi([x:y:z]) = [x^2:xy:xz:y^2:yz:z^2] \cong  
\begin{bmatrix}
    x^2 &xy  &xz\\
    xy  &y^2 &yz\\
    xz  &yz  &z^2\\
\end{bmatrix} 
\cong  
\begin{bmatrix}
    x\\y\\z\\
\end{bmatrix}
\begin{bmatrix}
    x &y &z\\
\end{bmatrix}.
$$ Hence, the image of $\phi$ are the "matrices" of rank one in $\PP^9_k$. 
These are the matrices whose minors all have determinant zero. 
So, the image of $\phi$ in $\PP^9_k$ is $$Z_h(ae-bd,af-cd,dh-eg,ei-fh,ah-bg,bi-ch,di-fg,af-cd,ai-cg).$$ 
Thus, the image of $\phi$ in $\PP^6_k$ is $$ Z_h(ae-b^2,bf-ec,ei-f^2,af-bc,bi-fc,ai-c^2)$$ 
which is closed in $\PP^6_k$. 

\end{document}