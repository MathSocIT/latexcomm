% pre-setting
\documentclass[12pt]{amsart}
\usepackage[toc,page]{appendix}
\usepackage{amsfonts}
\usepackage{setspace}
\usepackage{amssymb,amsthm,amsmath}
\usepackage[numbers,sort&compress]{natbib}
\usepackage{color}
\usepackage{mathtools}
\usepackage{graphicx}
\usepackage{tikz}
\usepackage{tikz-cd}
\usepackage{yhmath}
\usepackage{stackengine}
\stackMath

\linespread{0}
\hoffset -5pc
\renewcommand{\baselinestretch}{1.4}
\renewcommand{\textwidth}{40.5pc}
\renewcommand{\arraystretch}{1.5}
\setlength{\parskip}{0.2pc}

\theoremstyle{plain}
\newtheorem{theorem}{Theorem}[section]
\newtheorem{corollary}[theorem]{Corollary}
\newtheorem{lemma}[theorem]{Lemma}
\newtheorem{proposition}[theorem]{Proposition}
\newtheorem{lemmaa}[theorem]{Lemma A.}

\theoremstyle{definition}
\newtheorem{definition}[theorem]{Definition}
\newtheorem{conjecture}[theorem]{Conjecture}
\newtheorem{example}[theorem]{Example}
\newtheorem{remark}[theorem]{Remark}
\newtheorem*{note}{Note}
\newtheorem{case}{Case}[section]
\newtheorem*{question}{Question}
\newtheorem{axiom}[theorem]{Axiom}
\newtheorem{convention}[theorem]{Convention}
\newtheorem{notation}[theorem]{Notation}
\newtheorem{claim}{Claim}
\newtheorem{Rule}{Rule}

\newtheorem{innercustomgeneric}{\customgenericname}
\providecommand{\customgenericname}{}
\newcommand{\newcustomtheorem}[2]{%
  \newenvironment{#1}[1]
  {%
   \renewcommand\customgenericname{#2}%
   \renewcommand\theinnercustomgeneric{##1}%
   \innercustomgeneric
  }
  {\endinnercustomgeneric}
}

\newcustomtheorem{ctheorem}{Theorem}
\newcustomtheorem{clemma}{Lemma}
\newcustomtheorem{cclaim}{Claim}

\def\mb{\mathbb}
\def\bf{\textbf}
\newcommand{\lf}{\lfloor}
\newcommand{\rf}{\rfloor}
\newcommand{\avk}{\bf v_1, \bf v_2, \ldots, \bf v_k}
\newcommand{\awk}{\bf w_1, \bf w_2, \ldots, \bf w_k}
\newcommand{\avkw}{\bf v_1\ \bf v_2\ \ldots\ \bf v_k}
\newcommand{\aak}{\alpha_1, \alpha_2, \ldots, \alpha_k}
\newcommand{\abk}{\beta_1, \beta_2, \ldots, \beta_k}
\newcommand{\dsum}{\sum_{i=1}}
\newcommand{\soe}{\bf{Ax}=\bf 0}
\newcommand{\bew}{\begin{equation*}}
\newcommand{\eew}{\end{equation*}}	
\newcommand{\tu}{\textup}
\newcommand{\RA}{\implies}
\newcommand{\rA}{\rightarrow}
\newcommand{\LA}{\impliedby}
\newcommand{\lA}{\leftarrow}
\newcommand{\LRA}{\Leftrightarrow}
\newcommand{\lrA}{\leftrightarrow}
\newcommand{\univ}{\mathcal U}
\newcommand{\mc}{\mathcal}
\newcommand{\union}{\cup}
\newcommand{\inter}{\cap}
\newcommand{\diff}{\backslash}
\newcommand{\lsq}{\langle}
\newcommand{\rsq}{\rangle}
\newcommand{\dom}{\tu{dom}}
\newcommand{\ran}{\tu{ran}}
\newcommand{\uhra}{\upharpoonright}
\newcommand{\Ima}{\tu{Im}}
\newcommand{\pred}{\tu{pred}}
\newcommand{\FN}{\textup{\bf{FN}}}
\newcommand{\WO}{\textup{\bf{WO}}}
\newcommand{\ORD}{\textup{\bf{ORD}}}
\newcommand{\FOD}{\textup{\bf{FOD}}}
\newcommand{\C}{\textup{\bf{C}}}
\newcommand{\V}{\textup{\bf{V}}}
\newcommand{\otp}{\textup{otp}}
\newcommand{\lap}{\lessapprox}
\newcommand{\lnp}{\lnapprox}
\newcommand{\mf}{\mathfrak}
\newcommand{\what}{\widehat}
\newcommand{\lloc}[1]{L^{#1}_{loc}}
\newcommand{\supp}{\tu{Supp}}
\newcommand{\barint}[1]{\stackinset{c}{}{c}{}{-\mkern4mu}{\displaystyle\int_{#1}}}
\newcommand{\cemb}{\subset\subset}


\newcommand{\dint}[4]{\displaystyle\int_{#1}^{#2} \! #3 \, \operatorname{d} \! #4}
\newcommand{\dd}{\,\mathrm{d}}
\newcommand{\norm}[1]{\left\| #1 \right\|}
\newcommand{\abs}[1]{\left\lvert #1 \right\rvert}
\newcommand{\RR}{\mathbb{R}}
\newcommand{\bdf}{\begin{definition}}
\newcommand{\edf}{\end{definition}}

\DeclareMathOperator{\essinf}{essinf}
\DeclareMathOperator{\esssup}{esssup}
\DeclareMathOperator{\meas}{meas}
\DeclareMathOperator{\divv}{div}
\DeclareMathOperator{\spann}{span}



\begin{document}

\title{Suggested Solution for MA3220 Ordinary Differential Equations Final (AY22/23 Sem 1)}
\author{Written by Qi Fulin\\ Audited by Ikhoon Eom}



%%% ----------------------------------------------------------------------
%\begin{abstract}



%\end{abstract}
%%% ----------------------------------------------------------------------

\maketitle

%\textbf{Keywords}: {}
\subsection*{Question 1} \ 
\begin{remark}
	In the actual exam, no justification is needed for Question 1. Justifications presented here are just for readers' reference.
\end{remark}
(a) False.

Since it is a non-linear first-order equation, defining $f(t,y)=y^\frac15$, we just need to verify whether both $f$ and $f_y$ are both continuous in some open rectangle containing $(0,0)$. Clearly, $f_y=\frac15y^{-\frac45}$ is discontinuous at $x=0$, so no such rectangle exists. 

(b) True. 

We can either see it from the superposition principle of linear ODEs, or
\begin{align*}
	\frac{d^3}{dt^3}(y_1+2y_2)=y'''_1+2y'''_3=e^ty_1+2e^ty_2=e^t(y_1+2y_2).
\end{align*}

(c) True.

One may verify that $e^{-At}$ is indeed the inverse of $e^{At}$ for all $t\in\mb R$.

(d) True.

Since $(1,2)$ is a saddle point for the given autonomous system, we conclude that the corresponding linearised system near $(1,2)$ has two real eigenvalues of opposite signs, and $(1,2)$ is a critical point for the modified autonomous system. 

Given that $\lambda\in\mb R$ being an eigenvalue of $A$ implies $-\lambda\in\mb R$ being an eigenvalue of $-A$, we have the linearised system near $(1,2)$ corresponding to the modified autonomous system also has two real eigenvalues of opposite signs, so $(1,2)$ is indeed a saddle point for the second autonomous system.

(e) One such equation is $y'(t)=y^2$.

We have
\begin{align*}
	y^{-2}y'(t)=1&\RA -y^{-1}=t-C\\
	&\RA y=\frac{1}{C-t}.
\end{align*}
With the given initial condition, we have $C=1$, so the (unique) solution to the equation on some open interval containing $t=0$ is 
\begin{align*}
	y=\frac{1}{1-t},
\end{align*}
which goes to infinity at $t=1$.

(f) The suitable guess is $Y=t(At+B)(C\sin t+D\cos t)$.

Solving the characteristic equation $r^2+1=0$ gives us $r=\pm i$, so the general solution to the homogeneous solution is 
\begin{align*}
	y=A\sin t+B\cos t.
\end{align*}
Hence, the suitable guess is
\begin{align*}
	Y=t(At+B)(C\sin t+D\cos t).
\end{align*}
We need to multiply $(At+B)(C\sin t+D\cos t)$ by $t$ since otherwise, $BC\sin t+BD\cos t$ will solve the homogeneous equation.

(g) $\alpha=3$.

Define $f(x,y)=\alpha x^2y+xy^2$ and $g(x,y)=(x+y)x^2$. If the equation is exact, then 
\begin{align*}
	f_y=g_x\RA \alpha x^2+2xy=3x^2+2xy\RA \alpha=3.
\end{align*}

\subsection*{Question 2} \ 

(i) Rewrite the equation as $\displaystyle y'-t^{-1}y =t^3$, which is a first-order linear equation. Since $t^3$ is continuous on $\mb R$, $t^{-1}$ is continuous on $\mb R\backslash\{0\}$, and the initial condition is $y(1)=2$, by the existence and uniqueness theorem, the maximum interval where a solution is certain to exist is $(0,\infty)$.

(ii) The integrating factor is 
\begin{align*}
	e^{\int -t^{-1}\ dt}=t^{-1}.
\end{align*}

Multiplying both sides of $\displaystyle y'-t^{-1}y =t^3$ by $t^{-1}$, we have
\begin{align*}
	\left(t^{-1} y\right)'=t^2\RA t^{-1}y=\frac13t^3+C\RA y=\frac13t^4+Ct.
\end{align*}

From the initial condition $y(1)=2$, we have
\begin{align*}
	C=\frac53,
\end{align*}
so the solution to the IVP is
\begin{align*}
	y=\frac13t^4+\frac53t.
\end{align*}

Hence, the actual interval where the solution exists is $(-\infty,\infty)$.

\subsection*{Question 3} \ 

(i) Denoting the system as $\bf x'=\bf A\bf x$, we first try to find the eigenvalues of $\bf A$. The characteristic polynomial of $\bf A$ is
\begin{align*}
	\det(\bf A-\lambda\bf I)=\left|
	\begin{array}{cc}
		3-\lambda & -2\\
		1 & 1-\lambda
	\end{array}
	\right|=(3-\lambda)(1-\lambda)+2=\lambda^2-4\lambda+5,
\end{align*}
setting it to 0 giving us $\lambda=2\pm i$.

We now try to find an eigenvector corresponding to $\lambda=2+i$:
\begin{align*}
	\left(
	\begin{array}{cc}
		1-i & -2\\
		1 & -1-i
	\end{array}
	\right)
	\left(
	\begin{array}{cc}
		a\\
		b
	\end{array}
	\right)=\bf 0&\RA 
	\left(
	\begin{array}{cc}
		1 & -1-i\\
		0 & 0
	\end{array}
	\right)
	\left(
	\begin{array}{cc}
		a\\
		b
	\end{array}
	\right)=\bf 0\\
	&\RA \left(
	\begin{array}{cc}
		a\\
		b
	\end{array}
	\right)=b\left(
	\begin{array}{cc}
		1\\
		1
	\end{array}
	\right)+bi\left(
	\begin{array}{cc}
		1\\
		0
	\end{array}
	\right),\ b\in\mb R,
\end{align*}
so an eigenvector corresponding to $\lambda=2+i$ is $\displaystyle\binom11+i\binom 10$.

We then have a complex-valued solution
\begin{align*}
	\bf x(t)&=e^{2t}(\cos t+i\sin t)\left[\binom11+i\binom 10\right]\\
		&=e^{2t}\left[\binom{\cos t-\sin t}{\cos t}+i\binom{\sin t+\cos t}{\sin t}\right],
\end{align*}
so the general solution to the given system is
\begin{align*}
	\bf x(t)=e^{2t}\left[C_1\binom{\cos t-\sin t}{\cos t}+C_2\binom{\sin t+\cos t}{\sin t}\right].
\end{align*}

From the initial condition $\displaystyle \bf x(0)=\binom21$, we have
\begin{align*}
	C_1\binom11+C_2\binom10= \binom21\RA C_1=1,C_2=1,
\end{align*}
so the solution to the given IVP is
\begin{align*}
	\bf x(t)=e^{2t}\left[\binom{\cos t-\sin t}{\cos t}+\binom{\sin t+\cos t}{\sin t}\right]=e^{2t}\binom{2\cos t}{\cos t+\sin t}.
\end{align*}

(ii) The phase portrait is an anticlockwise outward spiral centred at $(0,0)$. Since the real part of the eigenvalue is $2>0$, we conclude that the phase portrait is a spiral source and is unstable.

\subsection*{Question 4} \ 

(i) We define $x_1=x$, $x_2=x'$. From the second order equation, we have
\begin{align*}
	\left\{
	\begin{array}{l}
		x_1'=x_2\\
		x_2'=-x_2-\alpha\sin x_1
	\end{array}
	\right.\iff \bf x'=\bf f(\bf x),
\end{align*}
which is a first-order system. When $\bf f(\bf x)=0$, we have
\begin{align*}
	\left\{
	\begin{array}{l}
		x_2=0\\
		-x_2-\alpha\sin x_1=0
	\end{array}
	\right.\RA\left\{
	\begin{array}{l}
		x_2=0\\
		\sin x_1=0
	\end{array}
	\right.\RA\left\{
	\begin{array}{l}
		x_2=0\\
		x_1=k\pi,k\in\mb Z.
	\end{array}
	\right.
\end{align*}
Hence, all critical points of the first order system are $(k\pi,0)$ for $k\in\mb Z$.

(ii) We define $\displaystyle \bf u(t)=\binom{x_1(t)}{x_2(t)}$.

Since we have
\begin{align*}
	\bf A=\bf f'(\bf 0)=\left.\left(
	\begin{array}{cc}
		0 & 1\\
		-\alpha\cos x_1&-1
	\end{array}
	\right)\right|_{x_1=x_2=0}=\left(
	\begin{array}{cc}
		0 & 1\\
		-\alpha & -1
	\end{array}
	\right),
\end{align*}
the corresponding linearised system near $(0,0)$ is 
\begin{align*}
	\bf u'(t)=\left(
	\begin{array}{cc}
		0 & 1\\
		-\alpha & -1
	\end{array}
	\right)\bf u(t)=\bf A\bf u(t).
\end{align*} 

(iii) The eigenvalues of $\bf A$ are the roots of
\begin{align*}
	p_\bf A(t)=\det(\bf A-\lambda \bf I)=-\lambda(-1-\lambda)+\alpha=\lambda^2+\lambda+\alpha,
\end{align*}
which are
\begin{align*}
	\lambda_1=\frac{-1+\sqrt{1-4\alpha}}{2},\lambda_2=\frac{-1-\sqrt{1-4\alpha}}{2}.
\end{align*}

When the critical point $(0,0)$ is a stable node, we have $\lambda_2<\lambda_1<0$, which then implies that
\begin{align*}
	0<\sqrt{1-4\alpha}< 1\RA 0<1-4\alpha<1\RA 0< \alpha<\frac14.
\end{align*}

\subsection*{Question 5} \ 

(i) From the question, we have
\begin{align*}\tag{1}
	y''+\frac1x y+\left(1-\frac4{x^2}\right)y=0.
\end{align*}

Since $\displaystyle \lim_{x\rA0}\frac1x$ does not exist, we conclude that $x=0$ is a singular point.

Since $\displaystyle\lim_{x\rA0}x\cdot \frac1x=\lim_{x\rA0}1=1$ and $\displaystyle \lim_{x\rA0}x^2\left(1-\frac4{x^2}\right)=\lim_{x\rA0}x^2-4=-4$, we conclude that $x=0$ is a regular singular point.


(ii) Let $\displaystyle y=\sum_{n=0}^\infty a_nx^{n+r}$ be the ansatz, so
\begin{align*}
	y'=\sum_{n=0}^\infty(n+r)a_nx^{n+r-1}\land 
	y''=\sum_{n=0}^\infty(n+r)(n+r-1)a_nx^{n+r-2}.
\end{align*}

From the question, we have
\begin{align*}\tag{2}
	&\sum_{n=0}^\infty(n+r)(n+r-1)a_nx^{n+r}+\sum_{n=0}^\infty(n+r)a_nx^{n+r}+\sum_{n=0}^\infty a_nx^{n+r+2}-\sum_{n=0}^\infty 4a_nx^{n+r}=0,
\end{align*}
so the indicial equation is
\begin{align*}
r(r-1)+r-4=0\RA r^2-4=0\RA r^2=4.
\end{align*}

The roots of the indicial equation are thus $r=\pm 2$.

(iii) From (2), we know $\{a_n\}_{n\in\mb Z^+_0}$ satisfies
\begin{align*}
	\left\{
	\begin{array}{ll}
		a_1=0 & \tu{if $n=1$;}\\
		(n^2+2nr+r^2-4)a_n+a_{n-2}=0 & \tu{if $n\geq2$.}\\
	\end{array}
	\right.\RA\left\{
	\begin{array}{ll}
		a_1=0 & \tu{if $n=1$;}\\
		a_n=\frac{-a_{n-2}}{n^2+2nr+r^2-4} & \tu{if $n\geq2$.}\\
	\end{array}
	\right.
\end{align*}

When $r=2$, we have $a_i=0$ for odd $i\in\mb Z^+$ and 
\begin{align*}
	a_2&=-\frac1{12}a_0,\\
	a_4&=-\frac1{32}a_2=\frac1{384}a_0.
\end{align*}
Following this trend, we see that $a_i$ is well-defined for all even $i\in\mb Z^+$ as well,  
so a series solution for the case of $r_1=2$ exists. The first three non-zero terms of the series solution are
\begin{align*}
	y=a_0x^2-\frac1{12}a_0x^4+\frac1{384}a_0x^6.
\end{align*} 

When $r=-2$, we have
\begin{align*}
	a_4&=\frac{1}{16-16+4-4}a_2=\frac10a_2,
\end{align*}
which is undefined. Hence, there is no series solution for the case of $r_2=-2$.

\subsection*{Question 6} \ 

(i) When $\lambda=0$, we have
\begin{align*}
	y''=0\RA y=Cx+D
\end{align*}
by integrating both sides twice with respect to $t$, where $C,D\in\mb R$ are arbitrary constants.

From the boundary conditions, we have $D=0$ and $C+D-C=0$, so all solutions to the BVP when $\lambda=0$ must be in the form of
\begin{align*}
	y=Cx,
\end{align*}
where $C\in\mb R$. Hence, there exists non-trivial solutions to the equation when $\lambda=0$, so $\lambda=0$ is indeed an eigenvalue.

(ii) The characteristic equation of $y''+\lambda y=0$ is $m^2+\lambda=0$, whose roots are
\begin{align*}
	m_1=\sqrt{-\lambda},m_2=-\sqrt{-\lambda}.
\end{align*}

Assume $\lambda>0$. We have $m_1,m_2$ are both complex, i.e., $m_1=i\sqrt\lambda,m_2=-i\sqrt{\lambda}$. The general solution to the equation is thus
\begin{align*}
	y(t)=A\cos{\sqrt\lambda t}+B\sin{\sqrt\lambda t}.
\end{align*}

From the boundary conditions, we have
\begin{align*}
	y(0)=0&\RA A=0\\
	y(1)-y'(1)=0&\RA A\cos\sqrt\lambda+B\sin\sqrt\lambda+\sqrt \lambda A\sin\sqrt\lambda-\sqrt\lambda B\cos\sqrt\lambda=0,   
\end{align*}
which suggest
\begin{align*}
	B\sin\sqrt\lambda-\sqrt\lambda B\cos\sqrt\lambda=0&\RA \sin\sqrt\lambda-\sqrt\lambda \cos\sqrt\lambda=0\\
	&\RA \sqrt{1+\lambda}\sin(\sqrt\lambda-\arctan\sqrt\lambda)=0\\
	&\RA \sqrt{\lambda_n}-\arctan\sqrt{\lambda_n}=n\pi, n\in\mb Z^+\tag{$*$}.
\end{align*}
Therefore, as long as $\lambda$ satisfies $(*)$, we can choose $B\in\mb R$ for free, which then implies that non-trivial solutions to the eigenvalue problem exist, i.e., $\lambda_n$ is an eigenvalue.

Since by definition, $\displaystyle -\frac\pi2<\arctan x<\frac\pi2$, from $(*)$, we have
\begin{align*}
	n\pi -\frac\pi 2<\sqrt{\lambda_n}<n\pi +\frac\pi 2\RA \left(n-\frac12\right)^2\pi^2<\lambda_n<\left(n+\frac12\right)^2\pi^2,
\end{align*}
which then gives us the desired result.

(iii) We define $\displaystyle \lsq f,g\rsq=\int_{0}^1fg\ dt$.

Let $f,g$ be non-trivial eigenfunctions corresponding to different eigenvalues $\lambda_1$ and $\lambda_2\in\mb R$. We have
\begin{align*}
\begin{split}
	\lsq -\lambda_1 f,g\rsq=\lsq f'',g\rsq&=\int_{0}^1f''g\ dt\\
	&=\left[f'g\right]_{0}^1-\int_{0}^1f'g'\ dt\\
	&=\left[f'g\right]_{0}^1-\lsq f',g'\rsq,
\end{split}\\
	\lsq  f,-\lambda_2g\rsq=\lsq f,g'\rsq&=\int_{0}^1fg''\ dt\\
	&=\left[fg'\right]_{0}^1-\int_{0}^1f'g'\ dt\\
	&=\left[fg'\right]_{0}^1-\lsq f',g'\rsq.
\end{align*}

From the boundary conditions, we have $f(0)=g(0)=0$, $f(1)=f'(1)$, and $g(1)=g'(1)$, which then implies that
\begin{align*}
	\lsq -\lambda_1 f,g\rsq=f(1)g(1)-\lsq f',g'\rsq,\\
	\lsq f, -\lambda_2g\rsq=f(1)g(1)-\lsq f',g'\rsq.
\end{align*}

This suggests that 
\begin{align*}
	\lsq -\lambda_1f,g\rsq=-\lambda_1\lsq f,g\rsq=-\lambda_2\lsq f,g\rsq=\lsq f,-\lambda_2g\rsq,
\end{align*}
which implies $\lsq f,g\rsq=0$ given that $\lambda_1\neq\lambda_2$. We therefore conclude that eigenfunctions corresponding to different eigenvalues must be mutually orthogonal on $[0,1]$.

\subsection*{Question 7} \ 

(i) Choose $p(t)=0$ and $q(t)=1$, so the equation becomes $y''+y=0$. We can check that 
\begin{enumerate}
	\item our choice of $p(t)$ and $q(t)$ satisfies the question requirements;
	\item $y(t)=\sin x$ is indeed a non-constant solution to the equation which has infinitely many zeros.
\end{enumerate}

(ii) Impossible.

Suppose this is possible for the sake of finding a contradiction. Let $y(t)$ be an arbitrary non-constant solution to the equation, and $a,b$ be two consecutive zeros of $y(t)$. 

Since both $p(t)$ and $g(t)$ are continuous on $\mb R$, we must have $y(t)$ is continuous on $\mb R$. Hence, by the extreme value theorem, $y(t)$ must have a local extrema in $(a,b)$.

If $y(t)$ has a local maximum $c$ in $(a,b)$, we have $y(c)>0$, $y'(c)=0$, and $y''(c )<0$. Since $q(c)<0$ by assumption, we have $q(c)y(c)<0$, which then implies that 
\begin{align*}
	0=y''(c)+q(c)y(c)<0,
\end{align*}
a contradiction.

Similarly, if $y(t)$ has a local minimum $d$ in $(a,b)$, we have $y(c)<0$, $y'(c)=0$, and $y''(c )>0$. Since $q(c)<0$ by assumption, we have $q(c)y(c)>0$, which then implies that 
\begin{align*}
	0=y''(c)+q(c)y(c)>0,
\end{align*}
a contradiction.

Therefore, we conclude that $y(t)$ has no local extrema in $(a,b)$, a contradiction to the extreme value theorem. We then conclude that it is impossible to have a solution with infinitely many zeros when $p(t)<0$ for all $t\in\mb R$.



%\begin{thebibliography}{99}
%\bibitem{}
%\end{thebibliography}
\end{document}