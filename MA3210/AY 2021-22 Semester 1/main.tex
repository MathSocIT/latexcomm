\documentclass{article}
\usepackage{lmodern}
\usepackage{amssymb,amsmath}
\usepackage[margin=1in]{geometry}
\usepackage{enumitem}
\title{MA3210 - Mathematical Analysis II Suggested Solutions}
\author{(Semester 1, AY2021/2022)}
\date{Written by: Chow Boon Wei\\Audited by: Chong Jing Quan}
\newcommand{\R}{\mathbb{R}}
\newcommand{\N}{\mathbb{N}}

\setlength\parindent{0pt}

\begin{document}
\maketitle
\subsection*{Question 1}
True. For any $x\in[0,1]$ $f(x)<g(x)$, so $0<g(x)-f(x)$. Define $h:[0,1]\to \R$ by $h(x)=g(x)-f(x)$. 
Since $f$ and $g$ are continuous, so is $h$. By the extreme value theorem, there exists a 
$x_0 \in [0,1]$ such that for any $x\in [0,1]$, we have $h(x)\geq h(x_0) = g(x_0)-f(x_0)>0$.
So, $\int_0^1 g - \int_0^1 f = \int_0^1 (g-f) = \int_0^1 h \geq h(x_0)(1-0) >0$. 
Finally, we have $\int_0^1 g > \int_0^1 f$.

\subsection*{Question 2}
\begin{enumerate}[label=\roman*)]
\item We have 
\begin{align*}
F'(h) &= \frac{d}{dh} \int_{-h}^h f(x+t)t dt \\
&= \frac{d}{dh} \int_{-h}^0 f(x+t)t dt + \frac{d}{dh} \int_{0}^h f(x+t)t dt \\
&= -\frac{d}{dh} \int_{0}^{-h} f(x+t)t dt + \frac{d}{dh} \int_{0}^h f(x+t)t dt \\
&= -f(x-h)h + f(x+h)h \\
&= ( f(x+h) - f(x-h) ) h. 
\end{align*}

\item We have 
\begin{align*}
    0=\lim_{h\to 0} \frac{1}{h^3} \int_{-h}^h f(x+t)t dt 
    &= \lim_{h\to 0} \frac{F(h)}{h^3} \\
    &= \lim_{h\to 0} \frac{F'(h)}{3h^2} \\
    &= \lim_{h\to 0} \frac{( f(x+h) - f(x-h) ) h }{3h^2} \\
    &= \lim_{h\to 0} \frac{ f(x+h) - f(x-h) }{3h}  \\
    &= \frac{2}{3}f'(x). 
\end{align*}
So, $f'(x) = 0$ for all $x\in \R$. Together with $f(0)=b$, we have $f(x)=b$ for all $b\in\R$.
\end{enumerate}
\subsection*{Question 3}

First, we have $\begin{pmatrix}
    f_x(a+1,b+1) &f_y(a+1,b+1)
\end{pmatrix} = 
\begin{pmatrix}
    2(a+1) &1\\
    b+1 &a+1
\end{pmatrix}$ 
since $f_x(x,y) = \begin{pmatrix}
    2x\\ y
\end{pmatrix}$ and $f_y(x,y) = \begin{pmatrix}
    1\\ x
\end{pmatrix}$. Let $A = 
    \begin{pmatrix}
        2(a+1) &1\\
        b+1 &a+1
    \end{pmatrix}$
Now, we have 
\begin{align*}
&\left|f(a+1+h,b+1+k) - f(a+1,b+1)
- A\begin{pmatrix}
    h\\ k
\end{pmatrix}\right|\\
&=
\left|
\begin{pmatrix}
    (a+1+h)^2+(b+1+k)\\
    (a+1+h)(b+1+k)
\end{pmatrix}
-
\begin{pmatrix}
    (a+1)^2+(b+1)\\
    (a+1)(b+1)
\end{pmatrix}
-
\begin{pmatrix}
    2(a+1) &1\\
    b+1 &a+1
\end{pmatrix}
\begin{pmatrix}
    h\\ k
\end{pmatrix}
\right|\\
&=
\left|
\begin{pmatrix}
    h(2a+2+h)+k\\
    (a+1)k+(b+1)h+hk
\end{pmatrix}
-
\begin{pmatrix}
    2(a+1)h+k\\
    (b+1)h + (a+1)k
\end{pmatrix}
\right|\\
&=
\left|
\begin{pmatrix}
    h^2\\
    hk
\end{pmatrix}
\right|\\
&=\sqrt{h^4+h^2k^2}=h\sqrt{h^2+k^2}.
\end{align*}
Therefore,
\begin{align*}
    \lim_{h \to 0,k \to 0}\frac{1}{\sqrt{h^2+k^2}}\left|f(a+1+h,b+1+k) - f(a+1,b+1)
    - A\begin{pmatrix}
        h\\ k
    \end{pmatrix}\right|
    &=\lim_{h \to 0,k \to 0}h=0
\end{align*}
which shows that $f$ is differentiable at $(a+1,b+1)$.
\subsection*{Question 4}
\begin{enumerate}[label=\roman*)]
    \item Since $f_1,f_2$ and $f_3$ are continuously differentiable, so is $f$. We have 
    $$
    f'(r,\theta,z) = 
    \begin{pmatrix}
        \cos\theta &-r\sin\theta &0\\
        \sin\theta &r\cos\theta &0\\
        0 &0 &1\\
    \end{pmatrix}.
    $$ So, $\det(f'(r,\theta,z)) = r$ and we have $\det\left(f'\left(b+1,\frac{\pi}{4},ab\right)\right) 
    = b+1 \neq 0$. Since $f$ is continuously differentiable, by the inverse function theorem, there 
    exists a neighborhood $U$ of $x=(b+1,\frac{\pi}{4},ab)$ such that $f$ is injective from $U$ to 
    $f(U)$ and $f^{-1}$ is continuously differentiable in $f(U)$. 
    
    \item We have $Id = (Id)' = [f(f^{-1}(y))]' = f'(f^{-1}(y)) (f^{-1})'(y)$. Hence, 
    $$ (f^{-1})'(y) = (f'(f^{-1}(y)))^{-1} = (f'(x))^{-1} =
    \begin{pmatrix}
        \frac{1}{\sqrt{2}} &-\frac{b+1}{\sqrt{2}} &0\\
        \frac{1}{\sqrt{2}} &\frac{b+1}{\sqrt{2}} &0\\
        0 &0 &1\\    
    \end{pmatrix}^{-1}
    =
    \frac{1}{b+1}\begin{pmatrix}
        \frac{b+1}{\sqrt{2}} &\frac{b+1}{\sqrt{2}} &0\\
        -\frac{1}{\sqrt{2}} &\frac{1}{\sqrt{2}} &0\\
        0 &0 &1\\    
    \end{pmatrix}. $$
\end{enumerate}

\subsection*{Question 5}
Since $|f^{(n)}(0)|\leq (a+1)|0|^{(n+1)}=0$, we have $f^{(n)}(0)=0$ for all $n\geq 0$. 
In particular, this means that $f(0)=0$. 
By Taylor's theorem, for each $n\geq 0$ and $x\in(-2-a,2+a)\backslash\{0\}$, we have 
\begin{align*}
    |f(x)| &= \left| \sum_{k=0}^n \frac{f^{(k)}(0)}{k!}x^k + \frac{f^{(n+1)}(c)}{(n+1)!}x^{n+1} \right| \\&
= \left| \frac{f^{(n+1)}(c)}{(n+1)!}x^{n+1} \right| \\& \leq \frac{(a+1)|c|^{n+2}}{(n+1)!}|x|^{n+1} 
\\& \leq \frac{(a+1)}{(n+1)!} (a+2)^{2n+3}  \to 0 \text{ as } n\to\infty 
\end{align*}

for some $ c\in (0,x) \text{ or } (x,0)$. So, $|f(x)|\leq 0$, and we have $f(x) = 0$.

\subsection*{Question 6}
\begin{enumerate}[label=\roman*)]
\item Define $g:\R \to \R$ by $g(x)=F(x)+\frac{x}{2}$. Then, $$g(-x)=F(-x)=\frac{-x}{e^{-x}-1}+\frac{-x}{2} 
= \frac{-2x}{2e^{-x}-2}+\frac{-xe^{-x}+x}{2e^{-x}-2} = \frac{-x-xe^{-x}}{2e^{-x}-2} 
= \frac{xe^x+x}{2e^x-2}$$ and $$g(x) = F(x)+\frac{x}{2} = \frac{x}{e^{x}-1}+\frac{x}{2} 
= \frac{2x}{2e^{x}-2}+\frac{xe^{x}-x}{2e^{x}-2} = \frac{xe^{x}+x}{2e^{x}-2}.$$ 
So, $g(x)=g(-x)$, which means that $g$ is an even function. Therefore, $0=g^{(2n+1)}(0)=F^{(2n+1)}(0)=B_{2n+1}$ for $n\geq 1$. (Note that the odd derivative of an even function is odd.)

\item By writing $T(x)= \sum_{n=0}^\infty \frac{F^{(n)}(0)}{n!}x^n = 
\sum_{n=0}^\infty \frac{B_n}{n!}x^n = 1 - \frac{x}{2} + \sum_{n=1}^\infty \frac{B_{2n}}{(2n)!}x^{2n}$, 
we see that $T$ converges when 
\begin{align*}
    &\lim_{n\to\infty}\left| \frac{B_{2n+2}/(2n+2)!x^{2n+2}}{B_{2n}/(2n)!x^{2n}} \right| \\
    &= \lim_{n\to\infty}\left| \frac{B_{2n+2}x^2}{B_{2n}(2n+1)(2n+2)} \right|\\ 
    &= \lim_{n\to\infty}\left| \frac{B_{2n+2}}{(-1)^n\sqrt{4\pi (n+1)} 
    \left(\frac{n+1}{\pi e}\right)^{2n+2} }
    \frac{(-1)^{n-1}\sqrt{4\pi n} \left( \frac{n}{\pi e}\right)^{2n} }{B_{2n}} 
    \frac{\sqrt{n+1} \left(\frac{n+1}{\pi e}\right)^{2n+2}}
    {\sqrt{n}  (2n+1)(2n+2)\left(\frac{n}{\pi e}\right)^{2n} }\right| x^2\\
    &= \lim_{n\to\infty}\left| 
    \frac{\sqrt{n+1} \left(\frac{n+1}{\pi e}\right)^{2n+2}}
    {\sqrt{n} (2n+1)(2n+2)\left( \frac{n}{\pi e}\right)^{2n} }\right| x^2\\
    &= \lim_{n\to\infty}\left| 
    \sqrt{\frac{n+1}{n}}
    \left(\frac{n+1}{n}\right)^{2n}
    \frac{ \left(\frac{n+1}{\pi e}\right)^2}
    {(2n+1)(2n+2) }\right| x^2\\
    &= \lim_{n\to\infty}e^2\left| \frac{1}{\pi^2 e^2}
    \frac{ \left(n+1\right)^2}
    {(2n+1)(2n+2) }\right| x^2
    = \frac{1}{4\pi^2} x^2<1.
\end{align*}
So, the radius of convergence is $2\pi$.

\item We have $F(x)(e^x-1) = x$. So, we have 
\begin{align*}
    x=F(x)(e^x-1) 
    &= \sum_{n=0}^\infty\frac{B_n}{n!}x^n \sum_{n=1}^\infty\frac{1}{n!}x^n\\
    &= x \sum_{n=0}^\infty\frac{B_n}{n!}x^n \sum_{n=0}^\infty\frac{1}{(n+1)!}x^{n}\\
    &= x \sum_{n=0}^\infty \sum_{k=0}^n \frac{B_k}{k!} \frac{1}{(n+1-k)!}x^{n}\\
    &= x \sum_{n=0}^\infty \sum_{k=0}^n \frac{1}{(n+1)!}\frac{(n+1)!}{k!(n+1-k)!} B_kx^{n}\\
    &= x \sum_{n=0}^\infty \sum_{k=0}^n \frac{1}{(n+1)!}
    \begin{pmatrix}
        n+1\\
        k
    \end{pmatrix} B_kx^{n}
\end{align*}
So, for $n\geq 1$, we have
$\sum_{k=0}^n \frac{1}{(n+1)!}
\begin{pmatrix}
    n+1\\
    k
\end{pmatrix} B_k = 0$
which means that 
$\sum_{k=0}^n 
\begin{pmatrix}
    n+1\\
    k
\end{pmatrix} B_k = 0$. Finally, this means that for $n\geq 2$, 
$\sum_{k=0}^{n-1} 
\begin{pmatrix}
    n\\
    k
\end{pmatrix} B_k = 0$
\end{enumerate}
\end{document}
