\documentclass[12pt]{article}
\usepackage{graphicx}
\usepackage{subfig}
\newcommand\mytab{\tab \hspace{+1cm}}
\usepackage{fancyhdr}
\usepackage{enumitem}
\usepackage{tikz-cd}
\usepackage{geometry}
\DeclareMathAlphabet{\mathcal}{OMS}{cmsy}{m}{n}
\usepackage[most]{tcolorbox}
 \geometry{
 a4paper,
 total={170mm,257mm},
 left=20mm,
 top=20mm,
 }
\usepackage[utf8]{inputenc}
\linespread{1.2}
\usepackage{tikz,lipsum,lmodern}
\usepackage{hyperref}
\newtcolorbox{mybox}{
enhanced,
boxrule=0pt,frame hidden,
borderline west={4pt}{0pt}{green!75!black},
colback=green!10!white,
sharp corners
}
\newtcolorbox{mybox2}{
enhanced,
boxrule=0pt,frame hidden,
borderline west={4pt}{0pt}{blue!75!black},
colback=blue!10!white,
sharp corners
}
\usepackage{amsmath,amsfonts,amsthm,bm} % Math packages
\usepackage[most]{tcolorbox}
\usepackage{amssymb}
\hypersetup{
    colorlinks=true,
    linkcolor=blue,
    filecolor=magenta,      
    urlcolor=cyan,
    pdftitle={Overleaf Example},
    pdfpagemode=FullScreen,
    }
\usepackage{amsmath}

\usepackage{natbib}
\usepackage{wrapfig}
\usepackage{mathtools}
\usepackage{url}
\usepackage{pax}
\usepackage{pdfpages}
\usepackage{graphicx}
\usepackage{mathptmx}
\usepackage{tikz}
\usepackage{blindtext}
\newcommand{\U}{\mathcal{U}}
\newcommand{\R}{\mathbb{R}}
\newcommand{\Z}{\mathbb{Z}}
\newcommand{\N}{\mathbb{N}}
\newcommand{\X}{\times}


\title{MA1100 AY24/25 Sem 2 Final}
\author{
  Solution by Malcolm Tan Jun Xi\\
  Audited by Ryan Loh Wei Xuan
}
\date{}


\begin{document}

\maketitle
\noindent\textbf{Question 1:} In each part, determine whether the statement is true or false. Write \textbf{True} or \textbf{False} as your answer in the box next to the statement. No justification is needed.
\begin{enumerate}[label=\alph*)]
    \item Let $P,Q$ and $R$ be statements. Then $(P\implies Q)\implies R$ and $P\implies(Q\implies R)$ are logically equivalent
    \item Let $P(x)$ and $Q(x)$ be predicates with free variable $x$ and universe $\mathcal{U}$. Then $(\forall x\in\mathcal{U})[P(x)\land Q(x)]$ and $\{(\forall x\in\mathcal{U})P(x)\}\land\{(\forall x\in\U) Q(x)\}$ are logically equivalent
    \item For any sets $A,B$ and $C$, if $A\subseteq B\cup C$, then $A\subseteq B$ or $A\subseteq C$.
    \item If $f:X\to Y$ is a function and $Z$ is a proper subset of $Y$ such that for all $x\in X$, $f(x)\in Z$, then $ Z$ is the range of $f$.
    \item For any sets $X,Y,Z$ and any functions $f:X\to Y$ and $g:Y\to Z$, if $g\circ f$ is surjective, then $f$ is surjective
    \item If $f:X\to Y$ is a function, then for all $A\subseteq X$, we have $f^{-1}[f[A]]=A$
    \item If $a,b\in\Z$ with $a$ not zero, then $gcd(a,b)=gcd(a,a+b)$
    \item If $X$ and $Y$ are uncountable sets, then there exists a bijection $f:X\to Y$
    \item There does not exist a surjection $g:\Z\to\Z\times\Z\times\Z$
    \item There are exactly 720 surjective functions with domain $\N_6$ and codomain $\N_6$
\end{enumerate}
\newpage
\noindent\emph{Solution:}\\
    (a) By truth table, we have
    \begin{table}[h!]
        \centering
        \begin{tabular}{|c|c|c|c|c|c|c|}\hline
            $P$ & $Q$ & $R$ & $P\implies Q$ & $Q\implies R$ & $(P\implies Q)\implies R$ & $P\implies(Q\implies R)$\\\hline
            T &  T&  T&  T&  T&  T& T\\\hline
            T &  T&  F&  T&  F&  F& F\\\hline
            T &  F&  T&  F&  T&  T& T\\\hline
            T &  F&  F&  F&  T&  T& T\\\hline
            F &  T&  T&  T&  T&  T& T\\\hline
             F&  T&  F&  T&  T&  \textbf{F}& \textbf{T}\\\hline
             F&  F&  T&  T&  T&  T& T\\ \hline
 F& F& F& T& T& \textbf{F}&\textbf{T}\\\hline
        \end{tabular}
    \end{table}\\
    Therefore the statement is \textbf{False}.\qed\\[2em]
    (b) For the forward direction, suppose for all $x\in \U$, we have $P(x)$ and $Q(x)$ to be true. As such $\{(\forall x\in\mathcal{U})P(x)\}\land\{(\forall x\in\U) Q(x)\}$ is also true.\\
    For the reverse direction, suppose for all $x\in\U$, $P(x)$ is true and for all $x\in\U$, $Q(x)$ is true. Therefore $(\forall x\in\mathcal{U})[P(x)\land Q(x)]$ is also true. Hence the statement is \textbf{True}\qed\\[2em]
    (c) Consider $A=\{1,2\}$, $B=\{1\}$ and $C=\{2\}$. In this case $B\cup C=\{1,2\}$ which implies that $A\subseteq B\cup C$ but neither $A\subseteq B$ or $A\subseteq C$. Thus the statement is \textbf{False}.\qed\\[2em]
    (d) Since $Z\subseteq Y$ and for all $x\in X$, we have $f(x)\in Z$, then by definition $R_f=\{f(x)\ |\ x\in X\}$ which implies that $Z=R_f$. Therefore, the statement is \textbf{True}.\qed\\[2em]
    (e) Consider $X=\{1\}$, $Y=\{a,b\}$ and $Z=\{\gamma\}$, then we can define the followings
    \begin{itemize}
        \item $f(1)=a$
        \item $g(a)=\gamma$ and $g(b)=\gamma$
    \end{itemize}
    This implies that $g\circ f(1)=\gamma$ is surjective, but $f$ is not surjective onto Y. Therefore the statement is \textbf{False}.\qed\\[2em]
    (f) Consider $A=\{1\}$ and $f(x)=x^2$, then we have
    \begin{itemize}
        \item $f(1)=1$
    \end{itemize}
    This shows that $f[A]=\{1\}$. However, $f^{-1}[f[A]]=\{1,-1\}\neq A$. Therefore the statement is \textbf{False}.\qed\\[2em]
    (g) Suppose we have $\gcd(a,a+b)$, by Bezout Lemma, we have $x,y\in\Z$ such that
    \begin{align*}
        \gcd(a,a+b)=ax+(a+b)y=ax+ay+by=a(x+y)+by
    \end{align*}
    Since $x+y$ is the sum of integers, by Bezout Lemma, we have
    \begin{align*}
        a(x+y)+by=\gcd(a,b)
    \end{align*}
    Therefore 
    \begin{align*}
        \gcd(a,a+b)=\gcd(a,b)
    \end{align*}
    The statement is \textbf{True}.\qed\\[2em]
    (h) Consider $X=\R$ and $Y=\mathcal{P}(\R)$. Both are uncountable but they do not have the same cardinality. hence there do not exist a bijection $f:\R\to\mathcal{P}(\R)$. The statement is \textbf{False}\qed\\[2em]
    (i) Consider Cantor's Tuple Bijection. Bijection exists Hence there exist a surjection. The statement is \textbf{False}. \\[2em]
    (j) Since $|\N_6|=6$, then the number of surjective functions is give by
    \begin{align*}
        6!=720
    \end{align*}
    Therefore the statement is \textbf{True}.\qed\\[2em]
    \noindent\textbf{Question 2:} Let $f:\Z\X\Z\to\Z\X\Z$ be defined by, for all $(m,n)\in\Z\X\Z$,
\begin{align*}
    f(m,n)=(2m-5n,m-2n)
\end{align*}
\begin{enumerate}[label=\roman*)]
    \item Prove that $f$ is injective
    \item Determine whether $f$ is invertible and find its inverse function if it exists. Justify your answers.
\end{enumerate}
\emph{Solution:}\\
(i) Let $(m_1,n_1),(m_2,n_2)\in\Z\X\Z$ such that we have
\begin{align*}
    f(m_1,n_1)&=f(m_2,n_2)\\
    (2m_1-5n_1,m_1-2n_1)&=(2m_2-5n_2,m_2-2n_2)
\end{align*}
By equality of ordered pairs, we have
\begin{align*}
    2m_1-5n_1=2m_2-5n_2\qquad\text{and}\qquad m_1-2n_1=m_2-2n_2
\end{align*}
From the first equation, we have
\begin{align*}
    2(m_1-m_2)=5(n_1-n_2)
\end{align*}
From the second equation, we have
\begin{align*}
    m_1-m_2=2(n_1-n_2)
\end{align*}
Substituting the second equation to the first equation yields
\begin{align*}
    4(n_1-n_2)&=5(n_1-n_2)\\
    n_1-n_2&=0\\
    n_1&=n_2
\end{align*}
Similarly from the second equation, we have 
\begin{align*}
    m_1-m_2&=0\implies m_1=m_2
\end{align*}
This shows that $(m_1,n_1)=(m_2,n_2)$ as such $f$ is injective.\qed\\[2em]
(ii) For $f$ to be invertible, $f$ has to be a bijection. Since we have proven that $f$ is injective, we will determine whether $f$ is surjective. $f$ is surjective if for all $(a,b)\in Z\X\Z$, there exist $(m,n)\in\Z\X\Z$ such that
\begin{align*}
    f(m,n)=(a,b)
\end{align*}
We can choose $(m,n)=(5b-2a,2b-a)$ such that we have
\begin{align*}
    f(m,n)=(2(5b-2a)-5(2b-a),5b-2a-2(2b-a))=(10b-4a-10b+5a,5b-2a-4b+2a)=(a,b)
\end{align*}
Therefore $f$ is surjective. Since $f$ is both injective and surjective, $f$ is bijective as such $f$ is invertible with the inverse function to be
\begin{align*}
    f^{-1}(m,n)=(5n-2m,2n-m)
\end{align*}\qed\\[2em]
\textbf{Question 3:} Prove that if $X$ and $Y$ are nonempty finite sets and $|X|<|Y|$, then there does not exist a surjection $f:X\to Y$\\[2em]
\emph{Solution:} Define $|X|=m$ and $|Y|=n$ where $m,n\in\Z$ and $m<n$. Since $|X|=m$ implies that $X$ has only $m$ elements, then set $R_f=\{f(x)\ |\ x\in X\}$ can have at most $m$ elements. Therefore $|f[X]|\leq m$.  Suppose for the sake of contradiction that there exist a surjection $f:X\to Y$, then $R_f=Y$ which implies $|R_f|=n$. Therefore $m=n$ which is a contradiction.\qed\\[2em]
\textbf{Question 4:} Let $R$ be an equivalence relation on the set $\N_7=\{1,2,3,4,5,6,7\}$ such that
\begin{itemize}
    \item $R$ has exactly 4 distinct equivalence classes;
    \item $|[1]|=|[2]|=1$
    \item $(3,4)\in\R$ and $(3,5)\not\in R$
    \item $[4]\not=[6]$ and $[6]\cap[7]\not=\emptyset$
\end{itemize}
\begin{enumerate}[label=\roman*)]
    \item List all the elements in each equivalence class of $R$
    \item List all the elements of $R$
\end{enumerate}
\emph{Solution:}\\
(i) We have
\begin{itemize}
    \item $[1]=\{1\}$
    \item $[2]=\{2\}$
    \item $[3]=[4]=\{3,4\}$
    \item $[5]=[6]=[7]=\{5,6,7\}$
\end{itemize}\qed\\[2em]
(ii) We have
\begin{align*}
    R=\{(1,1),(2,2),(3,3),(3,4),(4,3),(4,4),(5,5),(5,6),(5,7),(6,5),(6,6),(6,7),(7,5),(7,6),(7,7)\}
\end{align*}\qed\\[2em]
\textbf{Question 5:} Let $\sim$ be the relation on $\R^2$ defined by, for all $(a,b),(c,d)\in\R^2$,
\begin{align*}
    (a,b)\sim(c,d)\iff |a|+|b|=|c|+|d|
\end{align*}
\begin{enumerate}[label=\roman*)]
    \item Prove that $\sim$ is an equivalence relation.
    \item Give a geometrical description of the equivalence class $[(1,0)]$ as a subset of $\R^2$
    \item Find a bijection $g:\R^2/\sim\to[0,\infty)$.
\end{enumerate}
\emph{Solution:}\\
(i) Consider
\begin{align*}
    (a,b)\sim (a,b)\iff |a|+|b|=|a|+|b|
\end{align*}
which is trivially true. Hence $\sim$ is reflexive\\[2em]
Next Consider
\begin{align*}
    (a,b)\sim (c,d)\iff|a|+|b|=|c|+|d|\iff |c|+|d|=|a|+|b|\iff (c,d)\sim (a,b)
\end{align*}
 Therefore $\sim$ is symmetric\\[2em]
Lastly, we can consider $(a,b)\sim(c,d)$ and $(c,d)\sim (e,f)$ then we have
\begin{align*}
    (a,b)\sim (c,d)\iff|a|+|b|=|c|+|d|\qquad\text{and}\qquad (c,d)\sim (e,f)\iff |c|+|d|=|e|+|f|
\end{align*}
Note that
\begin{align*}
    |a|+|b|=|e|+|f|\iff (a,b)\sim (e,f)
\end{align*}
Therefore $\sim$ is transitive\\[2em]
In conclusion, since $\sim$ is reflexive, symmetric and transitive, $\sim$ is an equivalence relation\qed\\[2em]
(ii) We have
\begin{align*}
    [(1,0)]=\{(x,y)\in\R^2\ |\ |x|+|y|=1\}
\end{align*}
This is the set of all points in $\R^2$ whose Manhattan norm(magnitude) is 1. This describes a diamond centered at the origin with vertices at $(1,0), (0,1), (-1,0), (0,-1)$ which are the boundaries of the unit ball.\qed\\[2em]
(iii) We have
\begin{align*}
    [(a,b)]=\{(x,y)\in\R^2\ |\ |x|+|y|=|a|+|b|\}
\end{align*}
As such we can define
\begin{align*}
    g([(a,b)])=|a|+|b|
\end{align*}
If $(a,b)\sim (c,d)$, then $|a|+|b|=|c|+|d|$ which implies that $g([a,b])=g([c,d])$. As such the function is well defined.\\
Suppose $g([a,b])=g([c,d])$, then $|a|+|b|=|c|+|d|$ which implies $(a,b)\sim (c,d)$ thus $[a,b]=[c,d]$. Hence $g$ is injective\\
For any $r\in[0,\infty)$, we can choose $[(a,b)]=[(r,0)]$ such that $g([(r,0)])=|r|+|0|=r$. As such $g$ is surjective. \\[2em]
In conclusion, 
\begin{align*}
    g([(a,b)])=|a|+|b|
\end{align*}
is the bijection $g:\R^2/\sim\to[0,\infty)$\qed\\[2em]
\textbf{Question 6:} Let $c,m\in\Z^+$, and let $d=\gcd(c,m)$, $r=\frac{c}{d}$ and $s=\frac{m}{d}$
\begin{enumerate}[label=\roman*)]
    \item Prove that $r$ and $s$ are relatively prime
    \item Prove that if $a,b\in\Z$ and $ca\equiv cb\mod m $, then $a\equiv b\mod s$
\end{enumerate}
\emph{Solution:}\\
(i) By Bezout Lemma, we can consider
\begin{align*}
    \gcd(r,s)=rx+sy
\end{align*}
where $x,y\in\Z$. Then we have
\begin{align*}
    \gcd(r,s)=rx+sy=\frac{cx}{d}+\frac{my}{d}=\frac{cx+my}{d}=\frac{d}{d}=1
\end{align*}
Hence $r$ and $s$ are relatively prime.\qed\\[2em]
(ii) Since we have $ca\equiv cb\mod m$, then $m\ |\ ca-cb$. As such there exist $k\in\Z$ such that
\begin{align*}
    ca-cb&=mk\\
    rd(a-b)&=sdk\\
    r(a-b)&=sk\implies s\ |\ r(a-b)
\end{align*}
Since $s$ and $r$ are relatively prime, then $s\ |\ (a-b)$ which implies that $a\equiv b\mod s$\qed\\[2em]
\textbf{Question 7:} Let $\N=\{1,2,3,...\}$ denote the set of all positive integers and let $\mathcal{P}(\N)$ be the power set of $\N$. Let
\begin{align*}
    X=\{S\in\mathcal{P}(\N)\ |\ S\ \text{is finite}\}
\end{align*}
that is, $X$ is the set of all finite subsets of $\N$. Prove that $X$ is denumerable.\\[2em]
\emph{Solution:} Since $S$ is finite, then we can let $|S|=n$ where $n\in\N$. $X_n$ will then be the set of all subsets of $\N$ of size exactly $n$. Since it is known that $\N$ is countable and $X_n\subseteq \N$, then $X_n$ is also countable. Now define
\begin{align*}
    X=\bigcup_{n=0}^\infty X_n
\end{align*}
As such $X$ is a union of countable sets with implies that $X$ is countable. Since $n\in\N$ is infinite, then $X$ is countable infinite which implies that $X$ is denumerable.\qed\\[2em]
\textbf{Question 8:} Let $\R^+$ denote the set of all positive real numbers.
\begin{enumerate}[label=\roman*)]
    \item For each $n\in\Z^+$, let
    \begin{align*}
        A_n=\left(\frac{1}{n},\infty\right)=\left\{x\in\R\ |\ x>\frac{1}{n}\right\}
    \end{align*}
    Prove that
    \begin{align*}
        \bigcup_{n=1}^\infty A_n=\R^+
        \end{align*}
        \item Prove that if $B$ is a subset of $\R^+$ and $B$ is uncountable, then there exists $t\in\R^+$ such that the set
        \begin{align*}
            \{x\in B\ |\ x>t\}
        \end{align*}
        is uncountable.
\end{enumerate}
\emph{Solution:}\\
(i) For the forward inclusion, we can let $x\in\displaystyle{\bigcup_{n=1}^\infty A_n}$, then there exist $n\in\Z^+$ such that $x\in A_n$. Hence
\begin{align*}
    x>\frac{1}{n}>0\implies x\in \R^+
\end{align*}
Therefore
\begin{align*}
    \bigcup_{n=1}^\infty A_n\subseteq \R^+
\end{align*}
For the reverse inclusion, we can let $x\in\R^+$, then $x>0$. Note that since $\displaystyle{\lim_{n\infty}\frac{1}{n}=0}$, there exist $n\in\Z^+$ such that
\begin{align*}
    x>\frac{1}{n}>0\implies x\in A_n\qquad\text{for some }n\in\Z^+
\end{align*}
Therefore
\begin{align*}
    x\in \bigcup_{n=1}^\infty A_n\implies \R^+\subseteq\bigcup_{n=1}^\infty A_n
\end{align*}
In conclusion, we have proven
\begin{align*}
    \bigcup_{n=1}^\infty A_n=\R^+
\end{align*}\qed\\[2em]
(ii) Suppose for the sake of contradiction that for all $t\in\R^+$, the set $\{x\in B\ |\ x>t\}$ is countable and since $B\subseteq \R^+$, then
\begin{align*}
    B=B\cap \R^+=B\cap\left(\bigcup_{n=1}^\infty A_n\right)=\bigcup_{n=1}^\infty (B\cap A_n)
\end{align*}
Let $t=\frac{1}{n}$ and $n\in\Z^+$ such that
\begin{align*}
    B\cap A_n=\left\{x\in B\ |\ x>\frac{1}{n}\right\}
\end{align*}
From our assumption, $B\cap A_n$ is countable.Note that the union of all countable sets is countable, Hence $\bigcup_{n=1}^\infty (B\cap A_n)$ is countable. However, the equality suggest that $B$ is also countable which is a contradiction. Therefore there exists $t\in\R^+$ such that the set is uncountable.\qed
\end{document}
