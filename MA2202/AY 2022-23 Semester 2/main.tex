
\documentclass{article}
\usepackage{lmodern}
\usepackage{amssymb,amsmath}
\usepackage[margin=1in]{geometry}
\usepackage{enumitem}

\newcommand{\R}{\mathbb{R}}
\newcommand{\N}{\mathbb{N}}
\newcommand{\Z}{\mathbb{Z}}

\setlength\parindent{0pt}
\title{MA2202 - Algebra I Suggested Solutions}
\author{(Semester 2, AY2022/2023)}
\date{Written by: Xu Xuezhou\\Audited by: Clarence Chew Xuan Da\\Typeset by: Chow Yong Lam}

\begin{document}\maketitle
\subsection*{Question 1}

(i) $d = \text{gcd}(16287, 7031) = 89$. This is because:\begin{align*}
    16287 &= 7032\times2+2225\\
    7032 &= 2225\times3 + 356\\
    2225 &= 356\times6+89\\
    356 &= 89\times4 + 0
\end{align*}
(ii)
\begin{align*}
    d = 89 &= 2225-6\times356\\
    &= 2225-6\times(7031-3\times2225)\\
    &= 19\times2225 -6\times7031\\
    &= 19\times(16287-2\times7031)-6\times7031\\
    &= 19\times16287-44\times7031
\end{align*}
(iii) By (ii),
\begin{align*}
    19\times16287-44\times7031 &= d\\
    (-2)\times19\times16287 - (-2)\times44\times7031 &=-2d\\
    -38\times16287+88\times7031 &=-2d\\
    88\times7031 &= -2d (\text{mod } 16287)
\end{align*}Thus, we found a solution $x=88.$

\subsection*{Question 2}
\begin{align*}
	f = \left(
	\begin{array}{ccccccccc}
		1&2&3&4&5&6&7&8&9\\5&8&6&1&4&2&9&3&7
	\end{array}
	\right)
\end{align*}
(i) By cyclic notation, $f=(154)(2836)(79)$.

(ii) $O(f) = 4\times3=12$.

(iii)$f=(154)(2836)(79)=(15)(54)(28)(83)(36)(79)$. $f$ is a composition of six transpositions, so it is an even permutation.

\subsection*{Question 3}
Let $H$ be a subgroup of $(\mathbb Z,+)$. If $H=\{0\},$ then we set $d=0$ and we are done. 

Otherwise, $H$ has a nonzero element $x$. Since $H$ is a group, it contains $-x$ too. Hence, $H$ contains at least one non-negative integer, namely $|x|$. Let $d$ be the smallest positive integer in $H$, then $0<d<|x|$.

Claim 1: $d\mathbb{Z}\subseteq H$.

Since $H$ is a group, $-d\in H$. For a positive integer $a$, $ad\in H$ and $(-a)d\in H$. Hence, $H$ contains all multiples of $d$, i.e., $d\mathbb{Z}\subseteq H$.

Claim 2: $H\subseteq d\mathbb{Z}$.

Let $x\in H$. By division algorithm, $x = qd+r$ where $0\leq r<d$. Since $qd\in H$, $r = x-qd\in H$. This forces $r = 0$ so $x-qd = 0 \implies x=qd\in d\mathbb{Z}$. 

In conclusion, $H = d\mathbb{Z}$.

\subsection*{Question 4}
(i) Since $M\subseteq\mathbb{Z}/d\mathbb{Z}, h\in\mathbb{Z}/d\mathbb{Z}$ where $h$ is the smallest positive integer in $M$. Suppose $nh$ is the largest element in $M$, then $(n+1)h = 0\in M$. $0=(n+1)h\in M$. Therefore, $h$ divides $d$.

(ii) By (i), $M = \{0,h,2h,\cdots,nh\}$ and $(n+1)h = d$. Thus, $nh = d-h$, which makes  $M = \{0,h,2h,\cdots,d-h\}$

\subsection*{Question 5}
Let $(G,*)$ be a cyclic group with generator $g$.

(i) Suppose $G$ is an infinite group. Define $\phi:(G,*)\rightarrow(\mathbb{Z},+)$, $\phi(g^n)=n$. Define $\varphi:(\mathbb{Z},+)\rightarrow(G,*)$, $\varphi(n) = g^n$. Since $\phi\circ\varphi(n)=\phi(g^n)=n$ and $\varphi\circ\phi(g^n) =\varphi(n)=g^n$, $\phi$ is invertible. Also, let $g^a, g^b\in(G,*)$. \[ \phi(g^a*g^b)=\phi(g^{a+b}) = a+b = \phi(g^a)+\phi(g^b)\]
Thus, $\phi$ is homomorphic. Therefore, $(G,*)$ is isomorphic to $(\mathbb{Z},+)$.

(ii) Suppose $G$ is a finite group. Define $\phi:(G,*)\rightarrow(\mathbb{Z}/d\mathbb{Z},+)$, $\phi(g^n)=n$. Define $\varphi:(\mathbb{Z}/d\mathbb{Z},+)\rightarrow(G,*)$, $\varphi(n) = g^n$. Since $\phi\circ\varphi(n)=\phi(g^n)=n$ and $\varphi\circ\phi(g^n) =\varphi(n)=g^n$, $\phi$ is invertible. Also, let $g^a, g^b\in(G,*)$, and let $d$ be the order of $(G,*)$.
\[ \phi(g^a*g^b)=\phi(g^{a+b-kd}) = a+b -kd\in(\mathbb{Z},+)\]
where $k = 0$ if $a+b<d$ and $k=1$ otherwise. Thus, $\phi$ is homomorphic. Therefore, $(G,*)$ is isomorphic to $(\mathbb{Z}/d\mathbb{Z},+)$.

(iii) Let $M$ be a subgroup of $G$. If $G$ is infinite, let $H$ be a subgroup of $(\mathbb{Z},+)$. By Question 3, there exists a non-negative integer $d$ such that $H = d\mathbb{Z}$. Since $G$ is isomorphic to $\mathbb{Z}$, $M$ is isomorphic to $H$. Then, $M$ is cyclic. Similarly, if $G$ is finite, $M$ is isomorphic to $\{0,h,2h,\cdots,d-h\}$ which is cyclic.

\subsection*{Question 6}
Let $(G,*)$ be a group. Given $x\in G$, define $S_x = \{gxg^{-1}\in G : g\in G\}, Z_x = \{g\in G: gx=xg\}$. The set $S_x$ is called the \textit{conjugacy class} of $x$ and $Z_x$ is called the \textit{centralizer} of $x$.

(i) Suppose $y \in S_x$, then there exists $g\in G$ such that $y=gxg^{-1}\in G$. 

Let $g' \in G$. Then $g'g\in G$ and:
\[g'y(g')^{-1} = g'gxg^{-1}(g')^{-1} = g'gx(g'g)^{-1}\in G \implies S_x\subseteq S_y.\]
On the other hand, $x = g^{-1}yg$ and:
\[g'x(g')^{-1} =g'g^{-1}yg(g')^{-1} = g'g^{-1}y(g'g^{-1})^{-1}\in S_x\implies S_y\subseteq S_x.\]
Hence, $S_x=S_y$.

(ii) Suppose $S_x\cap S_y$ is non-empty for some $x$ and $y$ in $G$. Let $z\in S_x\cap S_y$, then $z\in S_x$ and $z\in S_y$. By (i), $S_z=S_x=S_y$.

(iii) It is noted that $G = \cup_{x\in G}S_x$. From the previous two parts, 
\[G = \bigsqcup_{x\in G}S_x\].

\subsection*{Question 7}
Assume $(G,*)$ is a finite group.

(i) Let $g=e\in G$. Then $e\in Z_x$ and $Z_x$ is not empty.

(ii) Since $Z_x$ is non-empty, let $g_1, g_2\in Z_x$. Then $g_1x=xg_1, g_2x=xg_2$. Since $(G,*) $ is finite, we only need to show $g_1g_2x=xg_1g_2$.
\begin{align*}
    x &= g_1xg_1^{-1};\\
    x &= g_2xg_2^{-1}\\&= g_1( g_2xg_2^{-1})g_1^{-1}\\&= (g_1g_2)x(g_1g_2)^{-1}\\
    g_1g_2x&=xg_1g_2
\end{align*}
Therefore, $g_1g_2\in Z_x$, and $Z_x$ is a subgroup of $G$.

(iii) Consider $G$ acting on $G$ by conjugation. Then $Z_x$ and $S_x$ are the stabilizer and orbit of $x$ respectively. By the Orbit-Stabilizer Theorem, the result follows.

(iv) The sum of the sizes of the disjoint $S_x$ sets is 25. The size of $S_e$ is 1. The size of $S_x$ must be 1, 5 or 25. Take the sum among all the sizes, and consider modulo 5. If no such element exists, the total sum would be 1 modulo 5 which cannot be 25 (contradiction). Thus, such an element must exist.

\subsection*{Question 8}
Let $N$ be a normal subgroup of the symmetric group $(S_n,\circ)$ where $n\geq5$. Let $M = N\cap A_n$ where $A_n$ is the alternating group.

(i) $M$ is non-empty since the identity is in both $N$ and $A_n$. If $a, b\in M$, then $ab^{-1}$ is in both $N$ and $A_n$, and hence in $M$. Thus, $M$ is a subgroup.

(ii) For $g\in A_n$, we have
\[gM = g(N\cap A_n) = gN\cap gA_n = Ng\cap A_ng = (N\cap A_n)g = Mg\].
Since left cosets are right cosets, $M$ is a normal subgroup in $A_n$.

(iii) Since $A_n$ is simple for $n\geq5$, it follows that $M = \{e\}$ or $M = A_n$. 

If $M=A_n$, $N$ contains $A_n$ so $N$ has index at most 2, so $N$ is $A_n$ or $S_n$. 

Otherwise, $M=\{e\}$. $N$ only has 1 even permutation. If $N$ contains an odd permutation $g$, then $gg=e$ so $g$ is of order 2, so it is a product of disjoint transpositions. If it is one transposition $(a,b)$, then since $N$ is normal, we can conjugate it to be $(c,d)$, multiplying together forming $(a,b)(c,d)\neq e$. If it has at least two transpositions $(a,b)(c,d), \cdots$, then we can conjugate it to $h = (a,c)(b,d)\cdots$, where $a,b,c,d$ are distinct, and $gh = (a,d)(b,c)\neq e$. Thus, $N$ has no odd permutation so it follows that $N=M=\{e\}$.
























\end{document}
